\chapter{Einleitung}

\glqq We have a clear vision – to create a world where every object - from jumbo jets to sewing needles – is linked to the Internet.\grqq \cite[Helen Duce, Seite 1]{iot2020}

Das \textit{Internet of Things} ist ein rasant wachsender Anwendungsbereich. Angetrieben von einer zunehmenden Akzeptanz digitaler Systeme und der günstigen Entwicklung der Baugröße, Leistung und Zuverlässigkeit und nicht zuletzt der sinkende Preis haben dazu geführt, dass digitale Systeme heute in allen Lebensbereichen anzutreffen sind \cite{iot2020, weiser1991}.\\

Es ist inzwischen selbstverständlich, Steuerungssysteme in der Industrie, zum Teil sogar schon im privaten Haushalt, miteinander zu vernetzen und im Netzwerk zugänglich zu machen. So ist es möglich, die vorhandenen Daten jederzeit auch extern z. B. via Mobiltelefon abzurufen. Heute sind viele Geräte und Programme sehr spezialisiert, so dass man in komplexen Arbeitszusammenhängen viele verschiedene Systeme braucht. Dadurch ist oft auch ein hoher finanzieller Aufwand erforderlich.

Gleichzeitig hat eine Annäherung der Hardware-Hersteller an die \glqq Hobby\grqq -Entwickler stattgefunden und Projekte wie Arduino und Co. haben den Entwicklungsaufwand eigener Hard- und Software erheblich reduziert. Damit wird es erheblich erleichtert, bei individuellen Fragestellungen selber individuelle Lösungen zu entwickeln. Hat man nun viele individuelle Lösungen gefunden und somit viele verschiedenartige Messdaten erhalten, besteht die Notwendigkeit, diese wiederum zusammenzuführen.


\section{Zielsetzung}
Aus dieser Tendenz entstand die Idee, die unterschiedlichen im Laboralltag anfallenden Messungen mit einem einzigen flexiblen System steuern und die Resultate gemeinsam verfügbar zu machen. 

Ziel dieser Arbeit ist es, ein Steuersystem für Messanwendungen zu entwickeln, das sich einfach konfigurieren lässt, flexibel in der Anwendung ist und gleichzeitig kostengünstig bleibt. Besonderes Augenmerk soll dabei auf der ausreichenden Verfügbarkeit verschiedener \gls{gpio} liegen. Insbesondere Pulsbreitenmodulation und Analog/Digital-Konverter sind für Messanwendungen interessant. Die anfallenden Messdaten sollen protokolliert werden und extern verwendbar sein. Die zu entwickelnde Applikation soll auf keinen bestimmten Anwendungsfall spezialisiert sein. Sie soll vielmehr dem Anwender die Möglichkeit geben, sich eine für sein jeweiliges Projekt passende Umgebung zusammenzustellen. Weiter soll das System ohne ständige Überwachung arbeiten können. Es soll möglich sein, eine  entfernte Messstation aufzubauen, die nach belieben via LAN, WLAN oder auch GSM konfiguriert und überwacht werden kann.\\

[TODO: GRAFIK]


\section{Definitionen}
In dieser Arbeit wird ein Singleboard-Computer des Typs \textbf{BeagleBone Black Rev. A5C} verwendet. Andere Versionen dieses Computers sind, sofern kompatibel, ebenfalls verwendbar, allerdings nicht getestet. Um eine gut Lesabrkeit zu ermöglichen ist mit \glqq BeagleBone\grqq ~im Folgenden immer diese Version gemeint.