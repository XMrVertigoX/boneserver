\chapter{Erweiterungsmöglichkeiten}
Im Folgenden Kapitel werden weitere Möglichkeiten erläutert, durch die der boneserver noch besser an bestehende Laborumgebungen angepasst werden kann oder die weiteres Potential des Konzeptes nutzbar machen.

\section{Zusätzliche Anpassungsoptionen}
Das Interface könnte durch zusätzliche Anpassungsoptionen erweitert werden. Eine Möglichkeit besteht darin, die eizelnen Kacheln mit individuellen Bezeichnungen zu versehen.

Der \textit{Interface Controller} kann hierzu dahingehend erweitert werden. Da im WebSocket Server die Pink-Konfiguration ohnehin in einer \gls{json}-Struktur gespeichet wird, können eventuelle Namen einfach hier gespeichert werden. Die Daten werden im Initialisierungsprozess der Seite automatisch an den Browser übertragen.

Im \textit{Init Script} kann die Weiterverarbeitung ähnlich dem Status der Kacheln implementiert werden (Abb. \ref{lst:fakeResponseMessage}).

\begin{figure}[H]
  \begin{lstlisting}
  for(pin in pins) {
    if (!pins[pin].active) {
      responseHandler.toggle({parameters: {pin: pin}, response: false});
    }
  }
  \end{lstlisting}
  \caption{Fake response message um geladene Kacheln vo verändern}
  \label{lst:fakeResponseMessage}
\end{figure}


\section{Höher aufgelöste A/D-Wandlung}
Der maximale Takt der digitalen Inputs und der \gls{adc} ist durch die kleinste Zeiteinheit in JavaScript (1ms) vorgegeben. Denkbar wäre eine höher aufgelöste Taktung in einer separaten Software zu implementieren. In C/C++ könnte eine wesendlich höhere Abtastrate erreicht werden. Diese Samples müssen dann zu mehreren in eine Response Message gepackt und wie bisher übertragen werden.\\

Der \textit{Diagram Controller} muss dahingehend erweitert werden, nicht nur einzelne Tupel zu verarbeiten (Abb. \ref{lst:insertTupel}), sondern oben beschriebene Frames. Je nach Abtastrate sollte auch über eine entsprechenden Downsampling für die Anzeige nachgedacht werden.

\begin{figure}[H]
  \begin{lstlisting}
  diagramCtrl.util.addValue = function(pin, data) {
    diagramCtrl[pin]['data'].push(data);
  }
  \end{lstlisting}
  \caption{Tupel in ein Diagramm einfügen}
  \label{lst:insertTupel}
\end{figure}


\section{Alternative Steuerungsbibliothek -- octalbonescript}
Die Bibliothek \textit{octalbonescript}\footnote{https://github.com/theoctal/octalbonescript}, ein bonescript Fork, löst laut API einige Probleme der bonescript Library. Diese könnte als ersatz für die bisher verwendete bonescript Library verwendet werden. Dadurch würden auch die Module \textit{gpioControl.js} und \textit{pwmControl.js} nicht mehr nötig sein, da diese nur Bypass-Funktionen für fehlerhafte Funktionen enthalten.


\section{Alternative Hardware}
Denkbar wäre eine Portierung des boneserver auf eine andere Hardware-Plattform. Die Vorraussetzung hierfür ist ein Linux-System mit einem Netzwerkzugang und entsprechendem Speicher.\\

Um auf einer anderen Hardware arbeiten zu können ist eine passende Bibliothek, wie die bonescript Library BeagleBone, nötig, um die Hardware zu steuern. Das Modul \textit{boneControl.js} die gesamte Steuerung und muss dahingehend erweitert werden, dass die enthaltenen Fälle (Abb. \ref{lst:exampleCase}) passende Funktionen aus ihrer Bibliothek starten.

\begin{figure}[H]
  \begin{lstlisting}
    case 'analogRead':
      var pin = parameters.pin;

      response = bonescript[request.type](pin);
      break;
  \end{lstlisting}
  \caption{Beispiel-Case aus dem Modul \textit{boneControl.js}}
  \label{lst:exampleCase}
\end{figure}


\section{Erweiterte Konfigurationsmöglichkeiten in das Interface integrieren}
Die erweiterten Konfigurationsmöglichkeiten, die bisher aus sicherheitsgründen nur über eine Konfigurationsdatei möglich waren, können is das Interface integriert werden. Das Authentifizierungssystem bietet hierfür die möglichkeite verschiedene Benutzer- und Administrator-Accounts einzurichten (vgl. Abb. \ref{lst:lighttpdhtdigest}). Dazu muss eine Administrationsseite geschrieben und in einem passenden Verzeichnis abgelegt werden. Die Zugriffsbeschränkungen können dann wie oben beschrieben separat für dieses Verzeichnis konfiguriert werden.\\

Externe Laufwerke und USB-Sticks sollten dafür automatisch mit passenden Rechten gemountet werden und auswählbar sein.