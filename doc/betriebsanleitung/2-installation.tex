\section{Installation}
Als Betriebssystem wird \href{http://archlinuxarm.org/}{Arch Linux ARM} verwendet, eine Portierung von Arch Linux für ARM-Prozessoren. Arch Linux ARM stellt auch ein spezielles package repository zur Verfügung.

\subsection{SD-Karte vorbereiten}
Auf der Homepage von Arch Linux ARM gibt es eine Installationsanleitung, die laufend aktualisiert wird. Die folgende Anleitung ist daher im Wesentlichen eine Übersetzung ausgehend von einem Linux als Host-System. Stattdessen kann auch die mitgelieferte {\AA}ngström Distribution verwendet werden, die mit dem BeagleBone ausgeliefert wird.

\paragraph{Voraussetzungen} sind die Softwarepakete \textit{dosfstools} und \textit{wget} sowie root-Rechte und eine Micro SD-Karte mit mindestens 2GB Speicherkapazität.

\begin{enumerate}

\item Finden Sie zunächst heraus, welcher Laufwerkspfad der vorgesehenen SD-Karte enspricht. Meist \textit{/dev/sd[a, b, ...]} oder \textit{/dev/mmcblk[0, 1, ...]}.

{\paragraph{\color{red} ACHTUNG} \textbf{\color{red} Überprüfen Sie Laufwerkspfade genau, bevor Sie mit der Installation beginnen, da sonst irreparable Schäden am Host-System auftreten können!}}

\item Starten Sie \emph{fdisk}, um die SD-Karte zu formatieren:

\begin{lstlisting}
fdisk /dev/sdX
\end{lstlisting}

\item Erstellen Sie eine neue Partitionstabelle und die nötigen Partitionen.\\
Dazu geben Sie nacheinander die folgenden Kommandos ein (jeweils mit \textit{enter} bestätigen):

\begin{longtabu} to \textwidth {
	X[1]
    X[3]}
\textbf{Kommando} & \textbf{Funktion}\\
o & Erzeugt eine neue Partitionstabelle\\
n, p, 1 & Erzeugt eine neue (n), primäre (p), erste (1) Partition\\
\textit{enter} & Bestätigt den Default-Wert für den ersten Sektor\\
+64M & +64M als letzten Sektor setzt die Partitionsgröße auf 64MByte\\
t, e & Ändert den Partitionstyp auf \textit{W95 FAT16 (LBA)}\\
a, 1 & Setzt das boot flag der ersten Partition (je nach \emph{fdisk}-Version wird die erste Partition automatisch ausgewählt, da nur eine zur Verfügung steht)\\
n, p, 2 & Erzeugt eine neue (n), primäre (p), zweite (2) Partition\\
2x \textit{enter} & Setzt die Default-Werte für den ersten und letzten Sektor der Partition\\
w & Schreibt Änderungen in die Partitonstabelle
\end{longtabu}

\item Formatieren der ersten Partition:

\begin{lstlisting}
mkfs.vfat -F 16 /dev/sdX1
\end{lstlisting}

\item Formatieren der zweiten Partition:

\begin{lstlisting}
mkfs.ext4 /dev/sdX2
\end{lstlisting}

\item Laden Sie den \textit{bootloader tarball} herunter und entpacken Sie ihn auf die erste Partition der SD-Karte:

\begin{lstlisting}
wget http://archlinuxarm.org/os/omap/BeagleBone-bootloader.tar.gz
mkdir boot
mount /dev/sdX1 boot
tar -xvf BeagleBone-bootloader.tar.gz -C boot
sync && umount boot
\end{lstlisting}

\item Laden Sie den \textit{rootfs tarball} herunter und entpacken Sie ihn auf die zweite Partition der SD-Karte (hierzu müssen Sie als \textit{root} eingeloggt sein, \emph{sudo} reicht in diesem Fall nicht):

\begin{lstlisting}
wget http://archlinuxarm.org/os/ArchLinuxARM-am33x-latest.tar.gz
mkdir root
mount /dev/sdX2 root
tar -xf ArchLinuxARM-am33x-latest.tar.gz -C root
sync && umount root
\end{lstlisting}

\item Stecken Sie die SD-Karte in den BeagleBone und halten Sie die Taste S2 gedrückt, um von der SD-Karte zu booten, während Sie die Power-Taste (S3) betätigen.\\
Wenn das System gestartet ist, können Sie sich auf der Kommandozeile oder via \emph{ssh} einloggen.\\

Benutzernahme/Passwort lautet \textbf{root/root}.

\end{enumerate}

Aus Sicherheitsgründen sollten Sie nach dem Systemstart als erstes das root-Passwort ändern:

\begin{lstlisting}
passwd root
\end{lstlisting}

Da man sich, außer zu Wartungszwecken, nicht am System anmelden muss, kann auf die Erstellung eines regulären Benutzers verzichtet werden.


\subsection{Installation im internen Speicher}

\paragraph{Hinweis} Der BeagleBone hat zwar eine eingebaute Uhr allerdings keine Batterie. Nach einem Neustart kann es daher passieren, dass die interne Uhr auf den Default-Wert zurück gesetzt wird. Überprüfen Sie mittels \emph{date} die aktuelle Systemzeit und aktualisieren diese gegebenenfalls via \texttt{ntpdate -u pool.ntp.org}

\begin{enumerate}
\item Um Arch Linux direkt auf der eMMC zu installieren, aktualisieren Sie zunächst das eben gestartete System und installieren die Pakete \emph{wget}, \emph{dosfstools} und \emph{ntp}.

\begin{lstlisting}
pacman -Syu wget dosfstools ntp
\end{lstlisting}

Das Paket \emph{ntp} stellt hierbei das Programm \emph{ntpdate} zur Verfügung (s. O.).

\item Der interne Speicher ist bereits korrekt partitioniert, folgen Sie daher nur den Schritten 4 bis 7. Die Partionen sind \textit{mmcblk1p1} bzw. \textit{mmcblk1p2} (s. O.).

\item Fahren Sie das System herunter und warten Sie bis alle LEDs erloschen sind.

\item Entfernen Sie die SD-Karte und starten das System erneut.
\end{enumerate}


\subsection{boneserver installieren}

\paragraph{Repository klonen}
\textit{boneserver} ist via \href{https://github.com/XMrVertigoX}{GitHub}\footnote{https://github.com/XMrVertigoX} verfügbar. Führen Sie dazu zunächt ein Systemupdate durch, um alle Pakete auf den neusten Stand zu bringen und installieren Sie das Paket \emph{git}. Anschließend klonen Sie das Repository nach \texttt{/opt}.

\begin{lstlisting}
pacman -Syu git
git -C /opt clone https://github.com/XMrVertigoX/boneserver.git
\end{lstlisting}

\paragraph{Installationsskript ausführen} Im root-Verzeichnis des Repositories befindet sich ein Skript, welches die weitere Installtaion übernimmt. Wechseln Sie dazu in das Verzeichniss und führen das Installationsskript aus.

\begin{lstlisting}
cd /opt/boneserver
./install.sh
\end{lstlisting}

Hierbei werden alle erforderlichen Pakete und Module installiert, die Konfigurationsdateien verlinkt sowie die Daemons installiert und gestartet.\\

Wenn das Skript fehlerfrei durchgelaufen ist, wird der BeagleBone automatisch neu gestartet und die Installation ist abgeschlossen.
