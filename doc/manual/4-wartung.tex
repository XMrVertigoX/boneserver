\section{Wartung}
Als Wartungssystem wird die oben erstellte SD-Karte verwendet. Dazu starten Sie den BeagleBone von der SD-Karte und führen die oben beschriebenen Installationsschritte aus.\\

Die Paketverwaltung unter Arch Linux heißt \emph{pacman}, über sie können neue Pakete aus den Repositories installiert bzw. aktualisiert werden.\\
Ein kurzer Auszug aus der man-page zu den hier verwendeten Parametern:\\

Synopsis: \texttt{pacman <operation> [options] [targets]}\\

\begin{longtabu} to \textwidth {X[1] X[4]}

\textbf{Parameter} & \textbf{Beschreibung}\\
\textit{Operations}\\
-S, -\--sync & Synchronize packages. Packages are installed directly from the ftp servers, including all dependencies required to run the packages.\\

\textit{Sync Options}\\
-c, -\--clean & Remove packages that are no longer installed from the cache as well as currently unused sync databases to free up disk space.\\
-u, -\--sysupgrade & Upgrades all packages that are out of date.\\
-y, -\--refresh & Download a fresh copy of the master package list from the server(s) [...]. This should typically be used each time you use \textit{-u} or \textit{-\--sysupgrade}.\\
\end{longtabu}


\subsection{Backup}
Da der interne Speicher des BeagleBone \glqq nur\grqq ~2GB beträgt, kann ohne größerem Zeitaufwand ein komplettes Speicherabbild erstellt werden. Dies hat den Vorteil, dass es beim Einspielen von Backups keine Kompatibilitätsprobleme auftreten können.

Im Ordner \glqq scripts\grqq ~sind zwei shell-Skripte, die diesen Vorgang vereinfachen: \textit{backup.sh} und \textit{restore.sh}. Dabei wird das Image automatisch mit \emph{gzip} komprimiert, um Speicherplatz zu sparen. Das restore-Skript verwendet diese Dateien, um das Speicherabbild wieder auf den BeagleBone zu kopieren.\\

Sollte die SD-Karte nicht genügend Speicherplatz zur Verfügung stellen, kann ein USB-Stick verwendet werden. Dazu einfach das USB-Laufwerk einhängen\footnote{Anleitungen hierzu gibt es in ausreichender Zahl im Internet} und das entsprechende Verzeichnis als Zielverzeichnis angeben.

\begin{lstlisting}
backup.sh [Zielverzeichnis]
\end{lstlisting}

Das Zielverzeichnis ist dabei optional. Wenn kein Parameter übergeben wird, erstellt das Skript automatisch eine Datei in der Form \textit{backup-[timestamp].img.gz} im aktuellen Verzeichnis.

\begin{lstlisting}
restore.sh <Quelldatei>
\end{lstlisting}

Die Quelldatei ist hier Vorraussetzung.

\paragraph{Hinweis} Die Skripte verwenden intern \emph{dd}, um eine bitweise Kopie der eMMC des BeagleBone anzufertigen. Zudem ist die Quelle bzw. das Ziel immer \textit{/dev/mmcblk1}. Daher sollten diese Skripte nur von der SD-Karte aus verwendet werden.


\subsection{System aktualisieren}
Arch Linux verwendet die Rolling-Release-Technik, ein System, bei dem es keine großen Upgrades des gesamten Betriebssystems gibt, sondern die Softwarepakete einzeln laufend aktualisiert werden.\\
Trotz umfangreicher Tests der Pakete kann es zu Inkompatibilitäten kommen. Dies ist wahrscheinlicher, je mehr Pakete gleichzeitig aktualisiert werden. Daher sollte, gerade wenn das System nur selten aktualisiert wird, vorher ein vollständiges Backup angelegt werden (s. o.).\\

Das System kann jederzeit via \emph{pacman} aktualisiert werden:

\begin{lstlisting}
pacman -Syu
\end{lstlisting}


\subsection{boneserver aktualisieren}
Um die boneserver-Software zu aktualisieren, aktualisieren Sie zunächt Ihre Kopie des git Repositories und führen das Installationsskript erneut aus. Pakete, die bereits installiert sind, werden dabei nicht erneut installiert.

\begin{lstlisting}
cd /opt/boneserver
git pull
./install.sh
\end{lstlisting}


\subsection{System bereinigen}
\emph{pacman} speichert bei jeder Aktualisierung die alten Pakete, um jederzeit auf frühere Versionen zurückgreifen zu können. Je nach Häufigkeit der Aktualisierung und gemessen an der Kapazität der eMMC, kann der Speicher schnell knapp werden. Daher können alte Pakete via pacman in zwei Stufen gelöscht werden:

\begin{lstlisting}
pacman -Sc 
\end{lstlisting}

Löscht alle Paketversionen nicht mehr installierter Pakete und

\begin{lstlisting}
pacman -Scc
\end{lstlisting}

löscht sämtliche nicht verwendeten Pakete.
