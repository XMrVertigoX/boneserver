\chapter{Zusammenfassung}
Im Rahmen dieser Arbeit wurde ein webbasiertes Steuersystem für Messanwendungen entwickelt und vorgestellt. Die Grundidee war, ein vereinfachtes, flexibles System anzubieten, das je nach Anwendungskontext individuell und ohne großen Aufwand angepasst werden kann. Hardware und Programmierung sollten hauptsächlich mit auf dem Markt vorhandenen Mitteln realisiert werden, um Entwicklungsaufwand und Kosten gering zu halten. Auch Rechenleistung und Energie sollten überschaubar bleiben. Von zentraler Bedeutung für das Messsystem allerdings ist, dass es über das Internet abruf- und steuerbar ist.

Die Wahl der Hardware fiel auf den BeagleBone Black, weil bei ihm das Verhältnis zwischen Rechenleistung, Schnittstellenumfang und Preis gegenüber den Konkurrenzsystemen am günstigsten ist. Gleichzeitig ist der Prozessor eine aktive Produktlinie von Texas Instruments, einem namhaften Hersteller für Mikroprozessoren, zudem gibt es bereits werksseitig eine Softwarebibliothek, die die rudimentäre Hardwaresteuerung übernimmt. Die Wahl des Betriebssystems fiel auf Arch Linux, ein minimales Linux System mit großen Anpassungsmöglichkeiten, da eine Eigenentwicklung den Rahmen dieser Arbeit nicht vorgenommen wurde und Linux Systeme im Embeddedbereich zurzeit einen \gls{de-facto-standard} darstellen.\\

Da die Bibliothek zur Steuerung der Hardware in JavaScript bzw. Node.js implemtiert ist, wurde zunächst versucht, möglichst viel der benötigten Funktionalität via JavaScript umzusetzen. Größtes Problem dabei war, dass die meisten Webserverlösungen in Node.js im Zusammenhang mit dem BeagleBone viel zu langsam arbeiten. Zudem ist es sehr aufwändig,  die vollständige Funktionalität eines Webservers selbst zu implementieren. Das gesamte System hätte mit root-Rechten laufen müssen, was bei Webservern aus Sicherheitsgründen grundsätzlich vermieden werden sollte. Hier wäre weiterer Aufwand durch gewissenhaftes Implementieren von Sicherheitsmechanismen entstanden. Dies war allerdings nicht Teil des Projekts. Aus diesen Gründen wurde sich für ein mehrstufiges System entschieden, bei dem ein regulärer Webserver Dokumente ausliefert und ein zweites System sich ausschließlich mit der Verwaltung der GPIO beschäftigt.

Dabei stellte sich die Frage, in welcher Weise Steuerbefehle sinnvoll an den BeagleBone übertragen und Antworten bzw. Messdaten zurück erhalten werden könnten. Die Lösung dieses Problems fand sich in WebSockets und \gls{json}. Mit deren Hilfe ist es möglich, Daten ohne großen Daten-Overhead strukturiert auszutauschen. Vorteilhaft ist ebenfalls, dass es bereits einige Implementierungen in Node.js in Form von Modulen gibt.\\

Das größte Problem wärend der Entwicklung bestand darin, dass die Bonescript Bibliothek durch ihr frühes Entwicklungsstadium noch einige Fehler, insbesondere in der Steuerung der Pulsbreitenmodulation, aufweist. Lange Versuche diese Fehler zu beheben oder zu umgehen haben dazu geführt, dass die Funktionalität selbst in Modulform separat implementiert wurde. Ziel ist es aber dennoch, diese Funktionaltät wieder in die Bibliothek zu verlagern, sobald die Probleme dort gelöst sind. 

Gegen Ende des Projekts bin ich auf eine alternative Bibliothek gestoßen (\textit{octalbonescript}), die ihrer Beschreibung nach die API der Bonescript-Bibliothek vollständig implementiert, dabei allerdings wesentliche Fehler behebt. Im Rahmen dieser Arbeit blieb leider keine Zeit, diese vielversprechende Alternative eingehend zu prüfen.\\

Das Ergebnis dieser Arbeit ist ein System, mit dem sich einfache Messumgebungen schnell erstellen lassen und das außer einer Netzwerkverbindung keiner weiteren Technik bedarf. Dabei ist es unerheblich, ob es sich bei dem Netzwerk um ein lokales Netzwerk, ein Intranet im Laborkomplex oder das Internet handelt. Da nur wichtig ist, dass das Netzwerk betriebssystemweit vorhanden ist, sind auch GSM- oder WLAN-Verbindungen kein Problem.

Diese Arbeit zeigt, dass es mit Hilfe von Webtechnologien wie HTML5 und WebSockets möglich ist Echtzeitmesssysteme zu realisieren, die existierende Hardware verwenden und dadurch wesentlich kostengünstiger sind als spezialisierte Systeme namhafter Hersteller.