\chapter{Erweiterungsmöglichkeiten}
Im folgenden Kapitel werden weitere Möglichkeiten erläutert, durch die der Boneserver noch besser an bestehende Laborumgebungen angepasst und weiteres Potential des Konzeptes nutzbar gemacht werden kann.

\section{Zusätzliche Anpassungsoptionen}
Das Interface könnte durch zusätzliche Anpassungsoptionen erweitert werden. Eine Mög\-lich\-keit besteht darin, die einzelnen Kacheln mit individuellen Bezeichnungen zu versehen.

Der \textit{Interface Controller} kann dahingehend erweitert werden. Da im WebSocket Server die Pin-Konfiguration ohnehin in einer \gls{json}-Struktur gespeichet wird, können eventuelle Namen einfach hier gespeichert werden. Die Daten werden im Initialisierungsprozess der Seite automatisch an den Browser übertragen.

Im \textit{Init Script} kann die Weiterverarbeitung ähnlich dem Status der Kacheln implementiert werden (Abb. \ref{lst:fakeResponseMessage}).

\begin{figure}[H]
\begin{lstlisting}
for(pin in pins) {
  if (!pins[pin].active) {
    responseHandler.toggle({parameters: {pin: pin}, response: false});
  }
}
\end{lstlisting}
\caption{Fake response message, um geladene Kacheln zu verändern}
\label{lst:fakeResponseMessage}
\end{figure}


\section{Höher aufgelöste A/D-Wandlung}
Der maximale Takt der digitalen Inputs und der \gls{adc} ist durch die kleinste Zeiteinheit in JavaScript (1ms) vorgegeben. Denkbar wäre es, eine höher aufgelöste Taktung in einer separaten Software zu implementieren. In C/C++ könnte eine wesentlich höhere Ab\-tast\-rate erreicht werden. Diese Samples können dann zu mehreren als Antwort übertragen werden.\\

Der \textit{Diagram Controller} muss dahingehend erweitert werden, nicht nur einzelne Wertepaare zu verarbeiten (Abb. \ref{lst:insertTupel}), sondern oben beschriebene Frames. Je nach Abtastrate sollte auch über eine entsprechende Unterabtastung für die Anzeige nachgedacht werden.

\begin{figure}[H]
\begin{lstlisting}
diagramCtrl.util.addValue = function(pin, data) {
  diagramCtrl[pin]['data'].push(data);
}
\end{lstlisting}
\caption{Tupel in ein Diagramm einfügen}
\label{lst:insertTupel}
\end{figure}


\section{Alternative Steuerungsbibliothek -- octalbonescript}
Die Bibliothek \textit{octalbonescript}\footnote{https://github.com/theoctal/octalbonescript}, ein bonescript Fork, löst laut API verschiedene Probleme der Bonescript-Bibliothek. Sie könnte als Ersatz für die bisher verwendete Bonescript-Bibliothek genutzt werden. Dadurch würden auch die Module \textit{gpioControl.js} und \textit{pwmControl.js} nicht mehr nötig sein, da sie nur Bypassfunktionen für fehlerhafte Funktionen enthalten.


\section{Alternative Hardware}
Denkbar wäre eine Portierung des Boneservers auf eine andere Hardwareplattform. Die Voraussetzung hierfür ist ein Linux System mit einem Netzwerkzugang und entsprechendem Speicher.\\

Um auf einer anderen Hardware arbeiten zu können, ist eine passende Bibliothek wie die Bonescript-Bibliothek BeagleBone, nötig, um die Hardware zu steuern. Das Modul \textit{boneControl.js} regelt die gesamte Steuerung und muss dahingehend erweitert werden, dass die enthaltenen Fälle (Abb. \ref{lst:exampleCase}) passende Funktionen aus ihrer Bibliothek starten.

\begin{figure}[H]
\begin{lstlisting}
case 'analogRead':
  var pin = parameters.pin;

  response = bonescript[request.type](pin);
  break;
\end{lstlisting}
\caption{Beispiel-Case aus dem Modul \textit{boneControl.js}}
\label{lst:exampleCase}
\end{figure}


\section{Erweiterte Konfiguration in das Interface integrieren}
Die erweiterten Konfigurationsmöglichkeiten, die bisher aus Sicherheitsgründen nur über eine Konfigurationsdatei möglich waren, können in das Interface integriert werden. Das Authentifizierungssystem bietet hierfür die Möglichkeit, verschiedene Benutzer- und Ad\-mi\-ni\-stra\-tor-""Accounts einzurichten (vgl. Abb. \ref{lst:lighttpdhtdigest}). Dazu muss eine Administrationsseite geschrieben und in einem passenden Verzeichnis abgelegt werden. Die Zugriffsbeschränkungen können dann, wie oben beschrieben, separat für dieses Verzeichnis konfiguriert werden. Externe Laufwerke und USB-Sticks sollten dafür automatisch mit passenden Rechten eingebunden werden und in einem Menü auswählbar sein.


\section{Weitere GPIO}
Neben den bereits verfügbaren GPIO sind noch einige weitere vorhanden, die standardmäßig für den HDMI-Ausgang verwendet werden. Da das System primär per Fernzugriff verwaltet werden soll, kann der Anschluss deaktiviert werden. Diese GPIO können über die \textit{whitelist.json} freigegeben werden.