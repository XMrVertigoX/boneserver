\documentclass[a4paper, oneside, 12pt, bibliography = totocnumbered]{scrbook}

\usepackage[english, ngerman]{babel}
\usepackage[utf8]{inputenc}
\usepackage[xindy, nonumberlist, toc, numberedsection]{glossaries}
\usepackage[onehalfspacing]{setspace}

\usepackage{
	datetime,
	eurosym,
	float,
	graphicx,
	epstopdf,
	hyperref,
	listings,
	longtable,
	lscape,
	pdfpages,
	rotating,
	scrhack,
	tabu,
	wrapfig
}

\hypersetup{
	colorlinks = true,
	linkcolor = black,
	urlcolor = black
}
		
\definecolor{backcolour}{rgb}{0.90, 0.90, 0.90}

\lstdefinestyle{default}{
	backgroundcolor = \color{backcolour},
	breaklines = true
	basicstyle = \tiny,
	columns = fullflexible
}

\renewcommand*{\glspostdescription}{}
\input{glossary.tex}
\makeglossaries

\lstset{style = default}

\bibliographystyle{unsrtdin}

\author{Caspar Friedrich}
\title{Webgestütztes GPIO Management am Beispiel des BeagleBone Black}

\begin{document}

%Titlepages
\documentclass[thesis.tex]{subfiles}
\begin{document}

% Deutsche Titelseite
\begin{titlepage}
\begin{center}

% FH-Logo
\includegraphics[width = \textwidth]{images/thesis/imp_rechts.eps}\\[3cm]

% BA + Thema
Bachelorarbeit Medientechnik\\[0.5cm]
{\sffamily \bfseries \Huge Webgestütztes GPIO Management\\[0.25cm]
am Beispiel des BeagleBone Black}\\[2cm]

% Info Student
vorgelegt von\\[0.5cm]
\textbf{Caspar Friedrich}\\[0.5cm]
Mat.-Nr. 11062078\\[0.5cm]

\vfill

% Unterer Teil der Seite
Erstgutachter: Prof. Dr. Klaus Ruelberg (Fachhochschule Köln)\\[0.5cm]
Zweitgutachter: Prof. Dr. Luigi Lo Iacono (Fachhochschule Köln)\\[0.5cm]
Dezember 2014

\end{center}
\end{titlepage}

\end{document}
\documentclass[thesis.tex]{subfiles}
\begin{document}

% Englische Titelseite
\begin{otherlanguage}{english}

\begin{titlepage}
\begin{center}

% FH-Logo
\includegraphics[width = \textwidth]{images/thesis/imp_rechts.eps}\\[3cm]

% BA + Thema
Bachelor Thesis\\[0.5cm]
{\sffamily \bfseries \Huge Webbased GPIO Management using\\[0.25cm]
the example of the BeagleBone Black}\\[2cm]

% Info Student
submitted by\\[0.5cm]
\textbf{Caspar Friedrich}\\[0.5cm]
Mat.-Nr. 11062078\\[0.5cm]

\vfill

% Unterer Teil der Seite
First Reviewer: Prof. Dr. Klaus Ruelberg (Cologne University of Applied Sciences)\\[0.5cm]
Second Reviewer: Prof. Dr. Luigi Lo Iacono (Cologne University of Applied Sciences)\\[0.5cm]
December 2014

\end{center}
\end{titlepage}

\end{otherlanguage}

\end{document}

%Overview
\input{thesis_overview_de.tex}
\newpage
\documentclass[thesis.tex]{subfiles}
\begin{document}

% Uebersichtsseite, englischer Abschnitt
\begin{otherlanguage}{english}

\begin{center}
	\textbf{Bachelor Thesis}
\end{center}

\noindent \textbf{Titel:} Webbased GPIO Management using the example of the BeagleBone Black\\

\noindent \textbf{Reviewers:}
\begin{itemize}
	\item Prof. Dr. Klaus Ruelberg (Cologne University of Applied Sciences)
	\item Prof. Dr. Luigi Lo Iacono (Cologne University of Applied Sciences)
\end{itemize}

\noindent \textbf{Abstract:} In this Bachelor Thesis I present a net-supported measuring system which provides a remote measuring station by help of the BeagleBone Black GPIOs. The outstanding feature of this system is its basic configuration, that has no defined use case and thus allows a flexible implementation in various measurement situations. Consequently, the system is economical and easyly applicable in almost any laboratory situation.\\

\noindent \textbf{Keywords:} BeagleBone, Webinterface\\

\noindent \textbf{Date:} {\longdate \today}

\end{otherlanguage}

\end{document}
\newpage

\tableofcontents

\part{Dokumentation}
\input{thesis_1_einleitung.tex}
\input{thesis_2_grundlagen.tex}
\input{thesis_3_konfiguration.tex}
\input{thesis_4_implementierung.tex}
\input{thesis_5_erweiterungsmoeglichkeiten.tex}
\documentclass[thesis.tex]{subfiles}
\begin{document}

\chapter{Zusammenfassung}
Im Rahmen dieser Arbeit wurde ein webbasiertes Steuersystem für Messanwendungen entwickelt und vorgestellt. Die Grundidee war, ein vereinfachtes, flexibles System anzubieten, das je nach Anwendungskontext individuell und ohne großen Aufwand angepasst werden kann. Hardware und Programmierung sollten hauptsächlich mit auf dem Markt vorhandenen Mitteln realisiert werden, um Entwicklungsaufwand und Kosten gering zu halten. Auch Rechenleistung und Energie sollten überschaubar bleiben. Von zentraler Bedeutung für das Messsystem allerdings ist, dass es über das Internet abruf- und steuerbar ist.

Die Wahl der Hardware fiel auf den BeagleBone Black, weil bei ihm das Verhältnis zwischen Rechenleistung, Schnittstellenumfang und Preis gegenüber den Konkurrenzsystemen am günstigsten ist. Gleichzeitig ist der Prozessor eine aktive Produktlinie von Texas Instruments, einem namhaften Hersteller für Mikroprozessoren, zudem gibt es bereits werksseitig eine Softwarebibliothek, die die rudimentäre Hardwaresteuerung übernimmt. Die Wahl des Betriebssystems fiel auf Arch Linux, ein minimales Linux System mit großen Anpassungsmöglichkeiten, da eine Eigenentwicklung den Rahmen dieser Arbeit nicht vorgenommen wurde und Linux Systeme im Embeddedbereich zurzeit einen \gls{de-facto-standard} darstellen.\\

Da die Bibliothek zur Steuerung der Hardware in JavaScript bzw. Node.js implemtiert ist, wurde zunächst versucht, möglichst viel der benötigten Funktionalität via JavaScript umzusetzen. Größtes Problem dabei war, dass die meisten Webserverlösungen in Node.js im Zusammenhang mit dem BeagleBone viel zu langsam arbeiten. Zudem ist es sehr aufwändig,  die vollständige Funktionalität eines Webservers selbst zu implementieren. Das gesamte System hätte mit root-Rechten laufen müssen, was bei Webservern aus Sicherheitsgründen grundsätzlich vermieden werden sollte. Hier wäre weiterer Aufwand durch gewissenhaftes Implementieren von Sicherheitsmechanismen entstanden. Dies war allerdings nicht Teil des Projekts. Aus diesen Gründen wurde sich für ein mehrstufiges System entschieden, bei dem ein regulärer Webserver Dokumente ausliefert und ein zweites System sich ausschließlich mit der Verwaltung der GPIO beschäftigt.

Dabei stellte sich die Frage, in welcher Weise Steuerbefehle sinnvoll an den BeagleBone übertragen und Antworten bzw. Messdaten zurück erhalten werden könnten. Die Lösung dieses Problems fand sich in WebSockets und \gls{json}. Mit deren Hilfe ist es möglich, Daten ohne großen Daten-Overhead strukturiert auszutauschen. Vorteilhaft ist ebenfalls, dass es bereits einige Implementierungen in Node.js in Form von Modulen gibt.\\

Das größte Problem wärend der Entwicklung bestand darin, dass die Bonescript Bibliothek durch ihr frühes Entwicklungsstadium noch einige Fehler, insbesondere in der Steuerung der Pulsbreitenmodulation, aufweist. Lange Versuche diese Fehler zu beheben oder zu umgehen haben dazu geführt, dass die Funktionalität selbst in Modulform separat implementiert wurde. Ziel ist es aber dennoch, diese Funktionaltät wieder in die Bibliothek zu verlagern, sobald die Probleme dort gelöst sind. 

Gegen Ende des Projekts bin ich auf eine alternative Bibliothek gestoßen (\textit{octalbonescript}), die ihrer Beschreibung nach die API der Bonescript-Bibliothek vollständig implementiert, dabei allerdings wesentliche Fehler behebt. Im Rahmen dieser Arbeit blieb leider keine Zeit, diese vielversprechende Alternative eingehend zu prüfen.\\

Das Ergebnis dieser Arbeit ist ein System, mit dem sich einfache Messumgebungen schnell erstellen lassen und das außer einer Netzwerkverbindung keiner weiteren Technik bedarf. Dabei ist es unerheblich, ob es sich bei dem Netzwerk um ein lokales Netzwerk, ein Intranet im Laborkomplex oder das Internet handelt. Da nur wichtig ist, dass das Netzwerk betriebssystemweit vorhanden ist, sind auch GSM- oder WLAN-Verbindungen kein Problem.

Diese Arbeit zeigt, dass es mit Hilfe von Webtechnologien wie HTML5 und WebSockets möglich ist Echtzeitmesssysteme zu realisieren, die existierende Hardware verwenden und dadurch wesentlich kostengünstiger sind als spezialisierte Systeme namhafter Hersteller.

\end{document}



\appendix
\part{Anhang}
\chapter{Betriebsanleitung}
\documentclass[manual.tex]{subfiles}

\begin{document}

\section{Hardware}
Dieses Handbuch ist für den \textbf{BeagleBone Black Revision A5C} (im Folgenden kurz als BeagleBone bezeichnet) geschrieben und getestet. Sofern nachfolgende oder vorangegangen Revisionen zu dieser kompatibel sind, sollte die Installation aber dennoch problemlos möglich sein.

\end{document}

\input{manual_2_installation.tex}
\input{manual_3_betrieb.tex}
\documentclass[manual.tex]{subfiles}

\begin{document}

\section{Wartung}
Als Wartungssystem wird die oben erstellte SD-Karte verwendet. Dazu starten Sie den BeagleBone von der SD-Karte und führen die oben beschriebenen Installationsschritte aus.\\

Die Paketverwaltung unter Arch Linux heißt \emph{pacman}, über sie können neue Pakete aus den Repositories installiert bzw. aktualisiert werden.\\
Ein kurzer Auszug aus der man-page zu den hier verwendeten Parametern:\\

Synopsis: \texttt{pacman <operation> [options] [targets]}\\

\begin{longtabu} to \textwidth {X[1] X[4]}

\textbf{Parameter} & \textbf{Beschreibung}\\
\textit{Operations}\\
-S, -\--sync & Synchronize packages. Packages are installed directly from the ftp servers, including all dependencies required to run the packages.\\

\textit{Sync Options}\\
-c, -\--clean & Remove packages that are no longer installed from the cache as well as currently unused sync databases to free up disk space.\\
-u, -\--sysupgrade & Upgrades all packages that are out of date.\\
-y, -\--refresh & Download a fresh copy of the master package list from the server(s) [...]. This should typically be used each time you use \textit{-u} or \textit{-\--sysupgrade}.\\
\end{longtabu}


\subsection{Backup}
Da der interne Speicher des BeagleBone \glqq nur\grqq ~2GB beträgt, kann ohne größerem Zeitaufwand ein komplettes Speicherabbild erstellt werden. Dies hat den Vorteil, dass es beim Einspielen von Backups keine Kompatibilitätsprobleme auftreten können.

Im Ordner \glqq scripts\grqq ~sind zwei shell-Skripte, die diesen Vorgang vereinfachen: \textit{backup.sh} und \textit{restore.sh}. Dabei wird das Image automatisch mit \emph{gzip} komprimiert, um Speicherplatz zu sparen. Das restore-Skript verwendet diese Dateien, um das Speicherabbild wieder auf den BeagleBone zu kopieren.\\

Sollte die SD-Karte nicht genügend Speicherplatz zur Verfügung stellen, kann ein USB-Stick verwendet werden. Dazu einfach das USB-Laufwerk einhängen\footnote{Anleitungen hierzu gibt es in ausreichender Zahl im Internet} und das entsprechende Verzeichnis als Zielverzeichnis angeben.

\begin{lstlisting}
backup.sh [Zielverzeichnis]
\end{lstlisting}

Das Zielverzeichnis ist dabei optional. Wenn kein Parameter übergeben wird, erstellt das Skript automatisch eine Datei in der Form \textit{backup-[timestamp].img.gz} im aktuellen Verzeichnis.

\begin{lstlisting}
restore.sh <Quelldatei>
\end{lstlisting}

Die Quelldatei ist hier Vorraussetzung.

\paragraph{Hinweis} Die Skripte verwenden intern \emph{dd}, um eine bitweise Kopie der eMMC des BeagleBone anzufertigen. Zudem ist die Quelle bzw. das Ziel immer \textit{/dev/mmcblk1}. Daher sollten diese Skripte nur von der SD-Karte aus verwendet werden.


\subsection{System aktualisieren}
Arch Linux verwendet die Rolling-Release-Technik, ein System, bei dem es keine großen Upgrades des gesamten Betriebssystems gibt, sondern die Softwarepakete einzeln laufend aktualisiert werden.\\
Trotz umfangreicher Tests der Pakete kann es zu Inkompatibilitäten kommen. Dies ist wahrscheinlicher, je mehr Pakete gleichzeitig aktualisiert werden. Daher sollte, gerade wenn das System nur selten aktualisiert wird, vorher ein vollständiges Backup angelegt werden (s. o.).\\

Das System kann jederzeit via \emph{pacman} aktualisiert werden:

\begin{lstlisting}
pacman -Syu
\end{lstlisting}


\subsection{boneserver aktualisieren}
Um die boneserver-Software zu aktualisieren, aktualisieren Sie zunächt Ihre Kopie des git Repositories und führen das Installationsskript erneut aus. Pakete, die bereits installiert sind, werden dabei nicht erneut installiert.

\begin{lstlisting}
cd /opt/boneserver
git pull
./install.sh
\end{lstlisting}


\subsection{System bereinigen}
\emph{pacman} speichert bei jeder Aktualisierung die alten Pakete, um jederzeit auf frühere Versionen zurückgreifen zu können. Je nach Häufigkeit der Aktualisierung und gemessen an der Kapazität der eMMC, kann der Speicher schnell knapp werden. Daher können alte Pakete via pacman in zwei Stufen gelöscht werden:

\begin{lstlisting}
pacman -Sc
\end{lstlisting}

Löscht alle Paketversionen nicht mehr installierter Pakete und

\begin{lstlisting}
pacman -Scc
\end{lstlisting}

löscht sämtliche nicht verwendeten Pakete.

\end{document}


\documentclass[thesis.tex]{subfiles}
\begin{document}

\chapter{Tabellen}

\begin{longtabu} to \textwidth {
X[1]
X[4]}
\textbf{Paketname} & \textbf{Kurzbeschreibung}\\
base-devel & Enthält die benötigten Entwicklertools\\
python2 & Python in der Version 2; wird für den Node Package Manager benötigt.\\
lighttpd  & Der Webserver (siehe Abschnitt \ref{subsec:Lighttpd}).\\
vsftpd & Ein FTP Server; wird verwendet um Messdaten ohne Webinterface automatisch abzurufen (siehe Abschnitt \ref{subsec:vsftpd}).\\
linux-headers-am33x-legacy & Linux Header Files; werden benötigt, um Device Tree Overlays zu kompilieren.\\
nodejs & JavaScript Engine (Siehe Abschnitt \ref{subsec:Node.js}).\\
wget & Freies Kommandozeilenprogramm, um Dateien aus dem Internet herunterladen zu können.\\
zsh & Alternative Kommandozeile (Z-Shell); hohe Anpassungsmöglichkeiten.\\
grml-zsh-config & Konfigurationsdateien für zsh.\\
wpa\_supplicant & Unterstützung für WPA-verschlüsselte WLANs.\\
pv & Pipe Viewer; wird von den Skripten \textit{backup.sh} und \textit{restpre.sh} verwendet um Speicherabbilder des BeagleBone zu erstellen.\\
dtc-git-patched-20130410-1 & Device Tree Compiler mit \textit{dynamic link}-Patch; wird von der bonescript library benötigt.\\
haproxy-1.5.3-1 & Der Proxyserver (siehe Abschnitt \ref{subsec:HAProxy}).\\

\caption{Liste der benötigten Zusatzpakete}\\
\label{tab:additionalPackages}\\
\end{longtabu}

\end{document}

\printglossary[title = {Glossar}]
\bibliography{literature.bib}
\documentclass[thesis.tex]{subfiles}
\begin{document}

\addchap{Eidesstattliche Erklärung}

\noindent Ich versichere hiermit, die vorgelegte Arbeit in dem gemeldeten Zeitraum ohne fremde Hilfe verfasst und mich keiner anderen als der angegebenen Hilfsmittel und Quellen bedient zu haben.

\vfill

\noindent Köln, den 28. November 2014\\[1cm]

\noindent Caspar Friedrich

\end{document}


\end{document}
