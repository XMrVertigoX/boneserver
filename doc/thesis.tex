\documentclass[a4paper, oneside, 12pt, bibliography = totocnumbered]{scrbook}

\usepackage[english, ngerman]{babel}
\usepackage[utf8]{inputenc}
\usepackage[xindy, nonumberlist, toc, numberedsection]{glossaries}
\usepackage[onehalfspacing]{setspace}

\usepackage{
	datetime,
	eurosym,
	float,
	graphicx,
	epstopdf,
	hyperref,
	listings,
	longtable,
	lscape,
	pdfpages,
	rotating,
	scrhack,
	tabu,
	wrapfig,
	subfiles
}

\hypersetup{
	colorlinks = true,
	linkcolor = black,
	urlcolor = black
}

\definecolor{backcolour}{rgb}{0.90, 0.90, 0.90}

\lstdefinestyle{default}{
	backgroundcolor = \color{backcolour},
	breaklines = true
	basicstyle = \tiny,
	columns = fullflexible
}

\renewcommand*{\glspostdescription}{}
\input{glossary.include}
\makeglossaries

\lstset{style = default}

\bibliographystyle{unsrtdin}

\author{Caspar Friedrich}
\title{Webgestütztes GPIO Management am Beispiel des BeagleBone Black}

\begin{document}

%Titlepages
\subfile{thesis_titlepage_de}
\subfile{thesis_titlepage_en}

%Overview
\subfile{thesis_overview_de}
\newpage
\subfile{thesis_overview_en}
\newpage

\tableofcontents

\part{Dokumentation}
\subfile{thesis_1_einleitung}
\subfile{thesis_2_grundlagen}
\subfile{thesis_3_konfiguration}
\subfile{thesis_4_implementierung}
\subfile{thesis_5_erweiterungsmoeglichkeiten}
\subfile{thesis_6_zusammenfassung}


\appendix
\part{Anhang}
\chapter{Betriebsanleitung}
\subfile{manual_1_hardware}
\subfile{manual_2_installation}
\subfile{manual_3_betrieb}
\subfile{manual_4_wartung}

\subfile{thesis_tabellen}
\printglossary[title = {Glossar}]
\bibliography{literature.bib}
\subfile{thesis_erklaerung}

\end{document}
