\documentclass[thesis.tex]{subfiles}
\begin{document}

% Uebersichtsseite, deutscher Abschnitt
\begin{center}
	\textbf{Bachelorarbeit}
\end{center}

\noindent \textbf{Titel:} Webgestütztes GPIO Management am Beispiel des BeagleBone Black\\

\noindent \textbf{Gutachter:}
\begin{itemize}
	\item Prof. Dr. Klaus Ruelberg (Fachhochschule Köln)
	\item Prof. Dr. Luigi Lo Iacono (Fachhochschule Köln)
\end{itemize}

\noindent \textbf{Zusammenfassung:} In der vorliegenden Bachelorarbeit wird ein netztwerk-gestütztes Messsystem vorgestellt, das mit Hilfe verschiedener GPIO des BeagleBone Black eine entfernte Messstationen ermöglicht. Die Besonderheit des Systems liegt darin, dass es als Basis ohne irgendeine Anwendungs-Spezialisierung besteht – also flexibel einsetzbar ist und gleichzeitig im Verhältnis zu heute gängigen Messsystemen sehr kostengünstig.  Es ist einfach und vielseitig konfigurierbar und kann browserbasiert über ein Netzwerk gesteuert und überwacht werden.\\

\noindent \textbf{Stichwörter:} BeagleBone, Webinterface\\

\noindent \textbf{Datum:} {\longdate \today}

\end{document}