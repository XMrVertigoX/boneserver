\documentclass[a4paper, oneside, 12pt, bibliography=totocnumbered]{scrbook}

\usepackage[english, german]{babel}
\usepackage[utf8]{inputenc}
\usepackage[xindy, nonumberlist, toc, numberedsection]{glossaries}
\usepackage[onehalfspacing]{setspace}

\usepackage{
  eurosym,
  listings,
  scrhack,
  pdfpages,
  float,
  lscape,
  graphicx,
  rotating,
  wrapfig,
  hyperref,
  tabu,
  longtable,
  datetime
}

\hypersetup{
  colorlinks = true,
  linkcolor = black,
  urlcolor = black
}
    
\definecolor{backcolour}{rgb}{0.90, 0.90, 0.90}

\lstdefinestyle{default}{
  backgroundcolor = \color{backcolour},
  breaklines = true
  basicstyle = \tiny,
  columns = fullflexible
}

\makeglossaries
\input{appendix/glossary.tex}

\lstset{style = default}

\bibliographystyle{unsrtdin}

\author{Caspar Friedrich}
\title{Webgestütztes GPIO Management am Beispiel des BeagleBone Black}


\begin{document}

%Titlepages
% Deutsche Titelseite
\begin{titlepage}
\begin{center}

% FH-Logo
\includegraphics[width = \textwidth]{images/imp_rechts.eps}\\[3cm]

% BA + Thema
Bachelorarbeit Medientechnik\\[0.5cm]
{\sffamily \bfseries \Huge Webgestütztes GPIO Management\\[0.25cm]
am Beispiel des BeagleBone Black}\\[2cm]

% Info Student
vorgelegt von\\[0.5cm]
\textbf{Caspar Friedrich}\\[0.5cm]
Mat.-Nr. 11062078\\[0.5cm]

\vfill

% Unterer Teil der Seite
Erstgutachter: Prof. Dr. Klaus Ruelberg (Fachhochschule Köln)\\[0.5cm]
Zweitgutachter: Prof. Dr. Luigi Lo Iacono (Fachhochschule Köln)\\[0.5cm]
\monthname ~\the\year

\end{center}
\end{titlepage}

% Englische Titelseite
\begin{otherlanguage}{english}

\begin{titlepage}
\begin{center}

% FH-Logo
\includegraphics[width = \textwidth]{images/imp_rechts.eps}\\[3cm]

% BA + Thema
Bachelor Thesis\\[0.5cm]
{\sffamily \bfseries \Huge Webbased GPIO Management using\\[0.25cm]
the example of the BeagleBone Black}\\[2cm]

% Info Student
submitted by\\[0.5cm]
\textbf{Caspar Friedrich}\\[0.5cm]
Mat.-Nr. 11062078\\[0.5cm]

\vfill

% Unterer Teil der Seite
First Reviewer: Prof. Dr. Klaus Ruelberg (Fachhochschule Köln)\\[0.5cm]
Second Reviewer: Prof. Dr. Luigi Lo Iacono (Fachhochschule Köln)\\[0.5cm]
\monthname ~\the\year

\end{center}
\end{titlepage}

\end{otherlanguage}


%Overview
% Uebersichtsseite, deutscher Abschnitt
\begin{center}
	\textbf{Bachelorarbeit}
\end{center}

\noindent \textbf{Titel:} Webgestütztes GPIO Management am Beispiel des BeagleBone Black\\

\noindent \textbf{Gutachter:}
\begin{itemize}
	\item Prof. Dr. Klaus Ruelberg (Fachhochschule Köln)
	\item Prof. Dr. Luigi Lo Iacono (Fachhochschule Köln)
\end{itemize}

\noindent \textbf{Zusammenfassung:} In der vorliegenden Bachelorarbeit wird ein netztwerk-gestütztes Messsystem vorgestellt, das mit Hilfe verschiedener GPIO des BeagleBone Black eine entfernte Messstationen ermöglicht. Die Besonderheit des Systems liegt darin, dass es als Basis ohne irgendeine Anwendungs-Spezialisierung besteht – also flexibel einsetzbar ist und gleichzeitig im Verhältnis zu heute gängigen Messsystemen sehr kostengünstig.  Es ist einfach und vielseitig konfigurierbar und kann browserbasiert über ein Netzwerk gesteuert und überwacht werden.\\

\noindent \textbf{Stichwörter:} BeagleBone, Webinterface\\

\noindent \textbf{Datum:} {\longdate \today}\newpage
\input{title/overview_en.tex}\newpage

\tableofcontents

\part{Dokumentation}
\chapter{Einleitung}

\glqq We have a clear vision – to create a world where every object - from jumbo jets to sewing needles – is linked to the Internet.\grqq \cite[Helen Duce, Seite 1]{iot2020}

Das \textit{Internet of Things} ist ein rasant wachsender Anwendungsbereich. Angetrieben von einer zunehmenden Akzeptanz digitaler Systeme und der günstigen Entwicklung von Baugröße, Leistung und Zuverlässigkeit und nicht zuletzt der sinkende Preis haben dazu geführt, dass digitale Systeme heute in allen Lebensbereichen anzutreffen sind \cite{iot2020, weiser1991}.\\

Es ist inzwischen selbstverständlich, Steuerungssysteme in der Industrie, zum Teil sogar schon im privaten Haushalt, miteinander zu vernetzen und im Netzwerk zugänglich zu machen. So ist es möglich, die vorhandenen Daten jederzeit auch extern z. B. via Mobiltelefon abzurufen. Heute sind viele Geräte und Programme sehr spezialisiert, so dass man in komplexen Arbeitszusammenhängen viele verschiedene Systeme braucht. Dadurch ist oft auch ein hoher finanzieller Aufwand erforderlich.

Gleichzeitig hat eine Annäherung der Hardwarehersteller an die \glqq Hobby\grqq -Entwickler stattgefunden und Projekte wie Arduino und Co. haben den Entwicklungsaufwand eigener Hard- und Software erheblich reduziert. Damit wird es erheblich erleichtert, bei individuellen Fragestellungen selber individuelle Lösungen zu entwickeln. Hat man nun viele individuelle Lösungen gefunden und somit viele verschiedenartige Messdaten erhalten, besteht die Notwendigkeit, diese wiederum zusammenzuführen.


\section{Zielsetzung}
Aus dieser Tendenz entstand die Idee, die unterschiedlichen im Laboralltag anfallenden Messungen mit einem einzigen flexiblen System zu steuern und die Resultate gemeinsam verfügbar zu machen. 

Ziel dieser Arbeit ist es, ein Steuersystem für Messanwendungen zu entwickeln, das sich einfach konfigurieren lässt, flexibel in der Anwendung ist und gleichzeitig kostengünstig bleibt. Besonderes Augenmerk soll dabei auf der ausreichenden Verfügbarkeit verschiedener \gls{gpio} liegen. Insbesondere Pulsbreitenmodulation und Analog/Digital-Konverter sind für Messanwendungen interessant. Die anfallenden Messdaten sollen protokolliert werden und extern verwendbar sein. Die zu entwickelnde Applikation soll auf keinen bestimmten Anwendungsfall spezialisiert sein. Sie soll vielmehr dem Anwender die Möglichkeit geben, sich eine für sein jeweiliges Projekt passende Umgebung zusammenzustellen. Weiter soll das System ohne ständige Überwachung arbeiten können. Es soll möglich sein, eine  entfernte Messstation aufzubauen, die nach Belieben via LAN, WLAN oder auch GSM konfiguriert und überwacht werden kann.\\

\vfill

\includegraphics[width = \textwidth]{images/Einleitung.eps}

\vfill

\section{Definitionen}
In dieser Arbeit wird ein Singleboard-Computer des Typs \textbf{BeagleBone Black Rev. A5C} verwendet. Andere Versionen dieses Computers sind, sofern kompatibel, ebenfalls verwendbar, allerdings nicht getestet. Um eine gute Lesabrkeit zu ermöglichen, ist mit \glqq BeagleBone\grqq ~im Folgenden immer diese Version gemeint.
\chapter{Grundlagen}

\section{Hardware}

\subsection{\glsentrylong{sbc}}
Ein \gls{sbc}, zu deutsch Einplatinenrechner, ist ein Computersystem, bei dem alle für die Verwendung nötigen Bauteile auf einer einzelnen Platine aufgebracht sind. Hierbei sind neben den essenziellen Komponenten wie Prozessor, RAM und ROM auch Controller für verschiedene I/O-Schnittstellen, Oszillatoren oder Co-Prozessoren verbaut. Single-Board Computer werden vor allem in der Industrie als Steuersysteme eingesetzt, da sie oft billiger und flexibler sind als fest verdrahtete Steuersysteme. Mit fortschreitender Miniaturisierung und steigender Leistungsfähigkeit finden \glspl{sbc} heute auch in alltäglichen Geräten wie Autos, Waschmaschinen oder Fernbedienungen Verwendung.\\

Technisch gesehen sind auch die ersten Heimcomputer wie der \emph{C64} oder \emph{Atari ST} Single-Board Computer, allerdings lassen sich diese ohne Ein- und Ausgeabegräte wie Maus, Tastatur, Bildschirm nicht sinnvoll nutzen und werden in der Regel nicht als solche bezeichnet.


\subsubsection{Schnittstellen}
Single-Board Computer verfügen, je nach Anwendungsgebiet, über eine Vielzahl verschiedener analoger und digitaler I/O-Schnittstellen.\\

\noindent Übliche Schnittstellen sind:

\begin{itemize}
\item \gls{gpio}, darunter digitale I/O, \gls{pwm} und \gls{adc}
\item \gls{uart}
\item \gls{spi}
\item \gls{i2c}
\end{itemize}

Über \gls{uart} ist eine Implementierung der verbreiteten RS232/422/485-Schnittstellen möglich und auch üblich.\\

Aktuelle (Entwickler-)Systeme haben in der Regel einen oder mehrere USB-Anschlüsse (sowohl Client als auch Host Ports sind üblich) oder zumindest einen JTAG-Port, was die Programmierung wesentlich vereinfacht. In der Regel verfügen leistungsstärkere Systeme auch über einen Grafikausgang, oft HDMI oder eine der Miniaturvarianten.


\subsection{\glsentrylong{soc}}
Eng verknüpft mit der Entwicklung der SBC ist das Konzept des \gls{soc}. Hierbei werden die meisten oben genannten Komponenten eines Systems direkt in einem einzelnen IC verbaut. Meist sind nur ROM und Controller für höhere Schnittstellen USB oder LAN (in manchen Fällen auch Grafik) extern angebunden.

Heutige Single-Board Computer mit einem SOC können sehr leistungsstark sein, sind als Mehrkernsystem aufgebaut und haben Taktraten von mehreren GHz. Diese Computer sind vom Design her stark an Desktop-Systeme angepasst und können mit einem vollwertigen Linux- oder Windows-System betrieben werden.

Gerade bei diesen leistungsstarken SOCs hat sich die ARM-Architektur durchgesetzt. 1983 als Nebenprojekt gegründet hatte die 32-Bit-Architektur bereits 2002 einen Marktanteil von fast 80\% \cite{stiller2002}.\\

\noindent \gls{sbc} lassen sich (sehr) grob in zwei Klassen unterteilen:

\begin{enumerate}
\item \textbf{Leistungsschwache Systeme}\\
Die Taktraten dieser Prozessoren liegen üblicherweise deutlich unter 50 MHz, in seltenen Fällen bis 100 MHz. Diese Mikrocontroller werden meist direkt programmiert und finden vor allem im low energy-Sektor Anwendung.
\item \textbf{Leistungsstarke Systeme}\\
Hier ligen die Taktraten meist im GHz-Bereich. Hauptanwendungsbereiche sind Mobilfunksysteme und embedded computing in der Industrie. Gerade im Mobilfunkbereich sind oft Mehrkernsysteme anzutreffen und es wird bis auf wenige Ausnahmen oberhalb eines Betriebssystems, meist Linux bzw. Android, programmiert.
\end{enumerate}


\subsection{BeagleBone Black}
Für diese Arbeit verwende ich einen BeagleBone Black Rev. A5C (im Folgenden BeagleBone), ein Entwicklerboard mit einem ARM\textregistered ~Cortex\texttrademark -A8 Prozessor (Single Core) von Texas Instruments.\\

\noindent Die wichtigsten Features:
\begin{itemize}
\item 1GHz Taktrate
\item 512MB DDR3 RAM
\item 2GB\footnote{4GB ab Rev. C} Onboard Flash Memory
\item 10/100 Mbit/s Ethernet
\item 69\footnote{27 GPIO sind ohne weitere Konfiguration direkt verfügbar} \gls{gpio} mit mehreren \gls{pwm}-Ausgängen und \gls{adc}-Eingängen
\item Verhältnismäßig geringer Preis von ca. \EUR{45}
\end{itemize}


\section{Betriebssysteme}
Da die Ressourcen des BeagleBone Black sehr begrenzt sind, wird für diese Arbeit ein schlankes Betriebssystem benötigt, welches nur wenig Speicher verbraucht und geringen Leistungs-Overhead verursacht. Für diesen Zweck gibt es spezielle Versionen der bekannten Betriebssysteme wie Microsoft Windows oder Linux sowie verschiedene \gls{unixoideBetriebssysteme}.


\subsection{Linux}
Linux hat den Vorteil, dass nahzu alle Software als Quellcode verfügbar ist und im Bedarfsfall angepasst werden kann. Zudem ist es üblich Lizenzen zu verwenden, die eine nicht kommerzielle Anwendung sowie Anpassungen kostenfrei zulassen.

Ein eigenes Linux zu entwickeln oder ein \gls{buildSystem} zu verwenden wäre aus Sicht der Performance sicherlich die beste Wahl und ist in der Industrie weitgehend üblich. Im Rahmen dieser Arbeit wird eine bereits bestehenede \gls{linux-distribution} verwendet, da es bereits einige sehr schlanke und für den BeagleBone angepasste \glspl{linux-distribution} gibt.


\subsection{Linux Distributionen}
\href{http://beagleboard.org/}{BeagleBoard.org} bietet für den BeagleBone Black zwei verschiedene Distributionen an: {\AA}ngström und Debian. Beide Distributionen haben Vor- und Nachteile, die weiter unten erläutert werden. Ein weiteres Projekt, welches sich unter Entwicklern großer Beliebtheit erfreut ist Arch Linux, das als Basis für diese Anwendung dienen soll.

\subsubsection{The {\AA}ngström Distribution}
The {\AA}ngström Distribution ist auf dem BeagleBone vorinstalliert und stellt die Hauptdistribution dar. Diese Distribution nutzt ein Build System und findet im Wesentlichen Anwendung bei Speichersystemen wie NAS oder FTP-Server, wichtigstes Feature ist daher der geringe Leistungs- und Speicherbedarf. Bei dieser Distribution muss allerdings nahezu jede Software selbst kompiliert und eingerichtet werden.

\subsubsection{Debian Linux}
Im Allgemeinen gilt Debian Linux als (rock-)stable und ist eine der verbreitetsten Distributionen. Zudem basieren einige weitere namhafte Distributionen auf Debian Linux. Stärke und gleichzeitig auch Schwäche dieser Distribution sind die langen und umfangreichen Softwaretests. Wenn ein Paket in den offiziellen \glspl{repository} verfügbar ist kann man zwar davon ausgehen, dass es fehlerfrei funktioniert und mit allen anderen angebotenen Paketen kompatibel ist. Allerdings liegt es dann meist nicht mehr in der aktuellen Version vor. Das kann gerade bei Software aus dem Bereich Netzwerk/Internet problematisch werden.

\subsubsection{Arch Linux}
Gegenüber den oben genannten Distributionen hat Arch Linux zwei wesentliche Vorteile: Zum einen gibt es eine (Sub-)Distribution speziell für ARM-Prozessoren, bei der das Basissystem mit ca. 500MB sehr schlank ist. Zum anderen stellt Arch Linux ARM ein sehr umfangreiches \gls{softwareRepository} mit hoher Aktualität zur Verfügung. Zusätzlich gibt es das \gls{aur}, ein freies \gls{repository} in dem jeder Nutzer seine Pakete einstellen kann. Sämtliche in diesem Projekt verwendete Software lässt sich entweder direkt via Paketverwaltung aus den offiziellen \glspl{repository} installieren, oder kann vom Anwender selber kompiliert werden.

Arch Linux ARM verwendet ein Rolling-Release-Konzept, ein System kontinuierlicher Softwareentwicklung, bei der Pakete separat aktuell gehalten und weiter entwickelt werden. Es gibt keine explizite Betriebssystemversion sondern sogenannte Snapshots. So ist es  wesentlich einfacher, das System aktuell und sicher zu halten. Zwar kann es durchaus passieren, dass die eingestellte Software nicht \glqq out of the Box\grqq ~funktioniert. Aber in der sehr aktiven Community hinter Arch Linux bekommt man relativ schnell Hilfe.

Die Kernelentwicklung schreitet derzeit sehr schnell voran, daher wird zusätzlich zur regulären Kernelentwicklung ein Legacy-Paket mit einer stabilen Version (3.8) gepflegt, um nach einem Update des Kernelpakets nicht erst die Kompatibilität wieder herstellen zu müssen.


\section{Webtechnologien}
In der vorliegenden Arbeit kommen eine Vielzahl unterschiedlicher Webtechnologien zum Einsatz. An dieser Stelle werden nur die Grundpfeiler dieser Arbeit beschrieben, um die Architektur des Projekts darzustellen. Die weitere Auswahl verwendeter Technologien wird im Zusammenhang mit der Implementierung in Kapitel 4 beschrieben.

\subsection{Webserver}
Als Webserver wird in der Regel eine Software bezeichnet, die verschiedene Dokumente an einen Client, z. B. ein Webbrowser, ausliefert. Oft wird auch die ganze Hardware, auf der ein Webserver betrieben wird, als Webserver bezeichnet. Dies ist dann der Fall, wenn der Rechner ausschließlich zu diesem Zweck betrieben wird, z. B. bei größeren Webseiten.

Auf Grund verschiedener Spezialisierungen und Ansätze gibt es heute eine große Anzahl verschiedener Webserver. Apache HTTP Server der gleichnamigen Firma und Microsofts IIS haben sich hier in den letzten Jahren deutlich von der Konkurenz absetzen können \cite{webserversurvey1014}. IIS kommt für diese Arbeit allerdings nicht in Frage, da er nicht für Linux angeboten wird.

\subsubsection{Lighttpd}
Lighttpd ist ein Webserver, der durch seine besonders ressourcenschonende Implementierung hervorsticht. Dieser Webserver ist darauf spezialisiert, statische Dokumente auszuliefern, was ihn besonders passend für dieses Projekt macht. Außerdem lässt er sich wesentlich einfacher konfigurieren und über Module erweitern, als andere weiter verbreitete Webserver \cite{krieg2009}.


\subsection{Node.js}
\label{subsec:Node.js}
Node.js ist ein JavaScript-Framwork, das auf Googles V8-JavaScript-Engine basiert. Es stellt eine serverseitige Umgebung zur Verfügung, über die sich komplexe Netzwerkanwendungen leicht programmieren lassen. Dabei stellt Node.js ein Modulkonzept vor, mit dem man relativ unkompliziert Programmteile auslagern oder weitere Funktionalität einbinden kann. In der Basisinstallation liefert Node.js bereits eine große Bandbreite an Modulen mit, sodass viele Anwendungen ohne zusätzliche Software umgesetzt werden können \cite{springer2013}.

\subsubsection{npm}
Der Node Package Manager (NPM) erweitert Node.js um ein Werkzeug, mit dem sich externe Module leicht herunterladen und einbinden lassen. Unter \href{https://www.npmjs.org/}{npmjs.org} können zudem weitere Informationen und in der Regel auch eine API der Module eingesehen werden. Dank einer großen Community lassen sich hier zu den meisten allgemeinen Problemen bereits einige Lösungsansätze finden.

\subsubsection{bonescript}
Ausschlaggebend für die Verwendung von Node.js ist das Modul \textit{bonescript} von Jason Kridner, Mitgründer der \href{http://beagleboard.org/}{BeagleBoard.org} Foundation. Dieses Modul steuert über eine übersichtliche API weite Teile der Hardware, sodass auch hier der Arbeitsaufwand deutlich verringert wird.

\subsection{WebSockets}
WebSockets stellen eine Möglichkeit dar, Daten zwischen Client, z. B. der Browser, und Server auszutauschen. Wie auch \gls{http} basieren WebSockets auf \gls{tcp}, sind dabei aber wesendtlich schneller. Im Gegensatz zu \gls{http} ist über WebSockets eine Vollduplex-Verbindung möglich. Die Metadaten sind zudem überschaubar, sodass hierbei wenig zusätzliche Netzwerklast entsteht. Ein weiterer Vorteil ist, dass WebSockets von der \gls{sameOriginPolicy} ausgeschlossen sind \cite{rfc6455}.
\input{documentation/3-konfiguration.tex}
\chapter{Implementierung}

\begin{figure}[ht]
\centering
\includegraphics[width = 0.8\textwidth]{documentation/images/components.eps}
\caption{Funktionale Komponenten}
\label{fig:functionalComponents}
\end{figure}

Das Webinterface besteht aus drei funktionalen Teilen. Der erste Teil ist der Webbrowser, der ausschließlich als Client dient. Zweitens besteht das Webinterface aus dem WebSocket Server, der die Steuerung der \gls{gpio} erledigt und die Daten für die Konfiguration der Weboberfläche liefert; drittens aus einem Webserver, der die Dokumente ausliefert. Um nach außen über einen einzigen Port zu kommunizieren, ist ein Proxy Server vorgelagert (Abb. \ref{fig:functionalComponents}), der den Datenverkehr an Hand des verwendeten Protokolls an den richtigen Server weiterleitet (Abb. \ref{fig:requestForwarding}).

\begin{figure}[ht]
\centering
\includegraphics[width = 0.5\textwidth]{documentation/images/request.eps}
\caption{Verarbeitungschema eines Requests am Proxy Server}
\label{fig:requestForwarding}
\end{figure}

\noindent Die Auslagerung das Proxy Servers hat folgende Vorteile:

\begin{enumerate}
\item Das \gls{ssl-offloading} wird von HAProxy automatisch vorgenommen.
\item Der Webserver ist nicht direkt zu erreichen. D. h. eventuelle Einbruchsversuche über Fehler in der HTTP-Implemtierung von Lighttpd werden bereits im Proxy Server abgefangen, da ungültige HTTP-Header nicht weiterverarbeitet werden.
\item Um die direkte Hardware-Steuerung zu übernehmen, muss der WebSocket Server mit root-Rechten betrieben werden. Da WebSocket Request nicht durch den Webserver geleitet werden, ist das System selbst bei einer Übernahme des Webservers weitegehend sicher. Zudem ist für den Betrieb keinerlei Kommunikation zwischen Webserver und WebSocket Server notwendig (Vgl. Abb. \ref{fig:pageloadSequence}).
\item Das Projekt gestaltet sich übersichtlicher, da die einzelnen Komponenten über virtuelle Netzwerkverbindungen kommunizieren und somit leichter zu warten sind.   
\item HAProxy ist im Feldeinsatz erprobt und kann angesichts der weiten Verbreitung auf namhaften Webseiten als ausreichend stabil angenommen werden \cite{kuehnast2014}, während die Proxy-Unterstützung in Lighttpd nur mäßig dokumentiert ist.
\end{enumerate}


\section{Datenaustausch}
Der gesamte Datenaustausch des Interfaces läuft über die WebSocket-Verbindung ab. Dabei beinhaltet jede Message den Key \textit{type}, mit dem jeweils im Browser bzw. Websocket Server entschieden wird, wie weiter zu verfahren ist. Das \textit{parameters}-Objekt enthält alle weiteren Informationen (Abb. \ref{lst:requestMessage}).\\

\begin{figure}[ht]
\begin{lstlisting}
{
  type: "getPinMode",
  parameters: {
    pin: "P9_33"
  }
}
\end{lstlisting}
\caption{Auszug aus einer Request Message}
\label{lst:requestMessage}
\end{figure}

Um Antworten richtig zuordnen zu können, werden Rückgabewerte der Funktionen in einem \textit{response}-Objekt an die ursprüngliche Messsage angehängt. Alle Daten, die zu dieser Antwort geführt haben, sind dann noch vorhanden und können über einen Response Handler verarbeitet werden.\\

\begin{figure}[ht]
\begin{lstlisting}
{
  type: "getPinMode",
  parameters: {
    pin: "P9_33"
  },
  response: {
    pin: "P9_33",
    name: "AIN4"
  }
}
\end{lstlisting}
\caption{Auszug aus einer Response Message}
\label{lst:responseMessage}
\end{figure}

Durch diese Technik ist gewährleistet, dass das Interface immer aktiv bleibt und nicht durch eine langsame Netzwerkanbindung blockiert wird. Gleichzeitig werden asynchron eingehende Messages zeitlich unabhängig verarbeitet.

\section{Seitenaufruf}
Um die Systembelastung durch das Webinterface möglichst gering zu halten, wird die Webseite dynamisch über Templates in weiten Teilen erst im Browser generiert. Dabei wird der Nutzer zunächst via HTTP Digest Authentication authentifiziert. Ist diese erfolgreich, werden die Doumente der Webseite an den Browser ausgeliefert.\\

Abbildung \ref{fig:pageloadSequence} zeigt schematisch, wie die Initialisierung der Seite abläuft. Das Diagramm zeigt horizontal die vier beteiligten Entitäten und vertikal, ähnlich einem Sequenzdiagramm, den zeitlichen Ablauf. Antworten sind dabei immer Event-basiert. Gestrichelte Linien zeigen Antworten tieferliegender Schichten und Protokolle. Sie sind nicht Teil dieser Implementierung und sollen den Request/Response-Verlauf verdeutlichen. Die Authentifizierung ist im Webserver implementiert und Teil der Kommunikation zwischen Webbrowser und Webserver \cite{rfc7235}. Sie ist daher nicht im Diagramm explizit dargestellt.

\begin{figure}[ht]
\centering
\includegraphics[width = \textwidth]{documentation/images/pageload.eps}
\caption{Sequezieller Ablauf bei einem Seitenaufruf}
\label{fig:pageloadSequence}
\end{figure}


\section{Interaktion}
Die Kommunikation mit dem WebSocket Server läuft vollständig asynchron ab. Es wird grundsätzlich keine direkte Antwort auf eine Anfrage erwartet, um bei eventuell schlechter Netzwerverbindung nicht das Interface zu blockieren. Abbildung \ref{fig:interaction} zeigt diesen Vorgang exemplarisch.

\begin{figure}[ht]
\includegraphics[width = \textwidth]{documentation/images/sendRequest.eps}
\caption{Verarbeitung einer User-Interaktion}
\label{fig:interaction}
\end{figure}


\section{WebSocket Server}
Der WebSocket Server ist vollständig in JavaScript/Node.js implementiert und modular aufgebaut. Alle notwendigen Dateien finden sich im Verzeichnes \textit{node/}.\\
Eine Besonderheit von JavaScript/Node.js ist es, dass sich Module grundsätzlich ähnlich wie ein Singleton Pattern verhalten. Dabei wird bei einem erneuten Aufruf von \textit{require()} keine neue Instanz erzeugt, sondern Referenzen auf die Erste übergeben.\\
So müssen bei asynchronen Funktionsaufrufen nicht alle benötigten Variablen übergeben werden und es wird sichergestellt, dass alle Funktionen auf denselben WebSocket zugreifen. Dies ist vor allem wichtig, da Intervalfunktionen, die via \textit{setInterval()} aufgerufen werden, keinen direkten Zugriff erlauben. Zudem sollen laufende Timer bei einem erneuten Besuch der Seite (oder eine reload) ihre Daten direkt wieder an den Browser senden.

\begin{figure}[ht]
\centering
\includegraphics[width = \textwidth]{documentation/images/wssDependencies.eps}
\caption{Abhängigkeiten der Module des WebSocket Server}
\label{fig:wssDependencies}
\end{figure}

Abbildung \ref{fig:wssDependencies} zeigt die Abhängigkeit zwischen den einzelnen Modulen. Im Kasten unter dem Modulnamen sind Abhängigkeiten dargestellt, die nicht Teil dieser Implementierung sind.


\subsubsection{index.js}
Die index.js stellt das Hauptdokument dar. Von hier aus wird der Server gestartet. Über den Aufruf \textit{require()} werden die Module des Servers geladen und die Event Listener für die WebSocket-Verbindung registriert.

Bei einem erfolgreichen Verbindungsaufbau wird der WebSocket im \textit{websocket}-Objekt referenziert und für alle anderen Module zugänglich gemacht.


\subsection{Models}

\subsubsection{settingsControl.js}
\begin{wrapfigure}[4]{r}{0.25\textwidth}
\vspace{-14pt}
\centering
\includegraphics[width = 0.25\textwidth]{documentation/images/apiSettingsControl.eps}
\end{wrapfigure}

Das Modul \textit{settingsControl.js} liest die Einstellungsdateien ein und stellt diese programmweit zur Verfügung. Das Modul nutzt dafür die Datei \textit{settings-default.json}, in der alle Basiseinstellungen enthalten sind. Diese Einstellungen werden dann von denen in der Datei \textit{settings.json} überschrieben, sofern das Dokument vorhanden ist. Bei den beiden Dateien handelt es sich um \gls{json}-Dateien nach rfc6455 \cite{rfc6455}.

Es werden nur Parameter überschrieben, die sowohl in den Default-Einstellungen als auch in den eigenen vorhanden sind. So ist sicherggestellt, dass keine Parameter versehentlich ins System gelangen, die nicht dafür vorgesehen sind. Desweiteren ist es möglich, nur die Einstellungen einzutragen, die geändert werden sollen.

\subsubsection{interfaceControl.js}
\begin{wrapfigure}[4]{r}{0.25\textwidth}
\vspace{-14pt}
\centering
\includegraphics[width = 0.25\textwidth]{documentation/images/apiInterfaceControl.eps}
\end{wrapfigure}

Dieses Modul generiert bei Bedarf eine Liste der verfügbaren Pins und deren Typen sowie deren letztmalige Interface-Konfiguration bezüglich aktiver und inaktiver Kacheln.\\

\noindent Zwei Dateien werden zur Generierung des interface-Objektes ausgelesen:

\begin{itemize}
  \item \textit{whitelist.json} enthält eine Liste der Pins mit Informationen über die Verwendbarkeit. Die Liste ist statisch und muss zur Software passen, sie wird vom Programm nicht verändert.
  \item \textit{interface.json} ist eine String-Version der letzten Interface-Konfiguration. Falls sie existiert, wird ein kombiniertes Objekt generiert um sicherzustellen, dass Informationen über das Interface nicht verloren gehen und eventuell zusätzliche Pins dennoch verfügbar sind.
\end{itemize}

Wenn die WebSocket-Verbindung geschlossen wird, wird das im Speicher befindliche Interface-Objekt in die Datei interface.json geschrieben und kann beim nächsten Aufruf der Seite erneut geladen werden.

\subsubsection{websocket.js}
\begin{wrapfigure}[4]{r}{0.25\textwidth}
\vspace{-14pt}
\centering
\includegraphics[width = 0.25\textwidth]{documentation/images/apiWebsocket.eps}
\end{wrapfigure}

Dieses Modul verwaltet die eigentliche WebSocket-Verbindung. Essenziell ist die Funktion \textit{write()}. Hierüber können die restlichen Module direkt auf den Socket schreiben. Dabei wird intern geprüft, ob der Socket noch offen ist. Wenn eine neue Verbindung hergestellt wurde, wird diese sofort wieder beschrieben.


\subsection{Controller-Module}
Ein weiterer Teil des WebSocket Servers besteht aus einer Reihe von Controller-Modulen, die verschiedene Steuerungen übernehmen.

\subsubsection{boneControl.js}
\begin{wrapfigure}[4]{r}{0.25\textwidth}
\vspace{-14pt}
\centering
\includegraphics[width = 0.25\textwidth]{documentation/images/apiBoneControl.eps}
\end{wrapfigure}

Dieses Modul stellt den Kern der Hardware-Ansteuerung dar. Eingehende WebSocket Requests werden hier über \textit{type} identifiziert und abgearbeitet. Hierbei wird über eine Switch-Case-Anweisung entschieden, was zu tun ist. Requests mit unbekanntem Typ werden mit einer Fehlermeldung im \textit{response}-Objekt an das Interface zurückgesendet (Abb. \ref{fig:wssResponseHandler}). Die API der \textit{bonescript} Library ist in dieser Liste weitgehend abgebildet. Daneben werden hier auch sämtliche Parameter für das Interface zusammengestellt und an den Browser gesendet.

\begin{figure}[ht]
\centering
\includegraphics[width = 0.5\textwidth]{documentation/images/wssResponseHandler.eps}
\caption{Schematische Arbeitsweise des Response Handlers}
\label{fig:wssResponseHandler}
\end{figure}


\subsubsection{timer.js}

\begin{wrapfigure}[4]{r}{0.25\textwidth}
\vspace{-14pt}
\centering
\includegraphics[width = 0.25\textwidth]{documentation/images/apiTimer.eps}
\end{wrapfigure}

Mit Hilfe dieses Moduls werden digitale und analoge Inputs verwaltet. Über eine Switch-Case-Anweisung wird geprüft, ob es sich um einen analogen oder digitalen Input handelt, und dann mit Hilfe von \textit{setInterval()} ein Timer gestartet, der zyklisch eine anonyme Callback-Funktion aufruft. Das zeitliche Interval wird dem Settings-Modul entnommen. Die Funktionen senden die Ergebnisse selbstständig an die Webseite. Fehlende Dateien, Links oder Ordner werden ebenfalls automatisch erstellt. Um den Timer wieder beenden zu können, wird die Funktion zusätzlich in einem lokalen Timer-Objekt registriert.

\paragraph{Digital Input} Um unnötiges Datenvolumen zu vermeiden prüft diese Funktion zunächst, ob sich der Pin-Status gegenüber der letzten Abfrage geändert hat. Nur wenn sich der Wert unterscheidet, wird er an das Interface gesendet (Abb. \ref{fig:wssTimerDigital}).

\begin{figure}[ht]
\centering
\includegraphics[width = 0.5\textwidth]{documentation/images/wssTimerDigital.eps}
\caption{Anonyme \textit{digitalRead}-Funktion}
\label{fig:wssTimerDigital}
\end{figure}

\paragraph{Analog Input} Die Funktion für den analogen Input produziert einen kontinuierlichen Datenstrom, um ein fließendes Diagramm auf der Webseite zu realisieren und speichert diese Daten parallel in einer standardkonformen CSV-Datei \cite{rfc4180} (Abb. \ref{fig:wssTimerAnalog}).

\begin{figure}[ht]
\centering
\includegraphics[width = 0.5\textwidth]{documentation/images/wssTimerAnalog.eps}
\caption{Anonyme \textit{analogRead}-Funktion}
\label{fig:wssTimerAnalog}
\end{figure}

\subsection{Bypass-Module}
Die Bonescript-Bibliothek weist in einigen Fällen Fehler oder fehlende Features auf. Intern arbeitet die Bibliothek mit \gls{devicetree} Overlays, wobei es nicht möglich ist diese wieder zu entladen. Dies ist vor allem wichtig, wenn \gls{pwm}-Ausgänge bockiert wurden, um einen zweiten synchronen Ausgang zu erhalten. Um diese Funktionalität zu implementieren, wurden zwei Bypass-Module geschrieben, die das digital I/O- und das PWM-Management übernehmen.

\subsubsection{gpioControl.js}
\begin{wrapfigure}[4]{r}{0.25\textwidth}
\vspace{-14pt}
\centering
\includegraphics[width = 0.25\textwidth]{documentation/images/apiGPIOControl.eps}
\end{wrapfigure}

Dieses Modul steuert die digitalen I/O direkt über das Dateisystem. Dazu werden die in der Bonescript-Bibliothek hinterlegten Pin IDs genutzt.\\

Das Modul stellt Funktionen zum Aktivieren und Deaktivieren der \gls{gpio} zur Verfügung sowie zum Lesen und Schreiben der Parameter. Parameter werden in Form eines \gls{json}-Objektes übergeben und sind jeweils optional.

\subsubsection{pwmControl.js}
\begin{wrapfigure}[4]{r}{0.25\textwidth}
\vspace{-14pt}
\centering
\includegraphics[width = 0.25\textwidth]{documentation/images/apiPWMControl.eps}
\end{wrapfigure}

Die API dieses Moduls verhält sich analog zu dem der \gls{gpio}. Intern werden die Device Tree Overlays der Bonescript-Bibliothek verwendet. Diese werden per Filesystem über den \gls{capemgr} geladen.

\section{Website} Bei der Implementierung steht die Funktionalität im Fokus. Daher wurde so viel wie möglich mit bereits existierenden Bibliotheken gearbeitet, um Design und Darstellung umzusetzen. Diese Bibliotheken basieren intern auf \gls{html}, \gls{css} und \gls{js}.\\
Die Website selbst besteht daher nur aus einem einzigen \gls{html}-Dokument, in dem die Basisstruktur der Seite definiert wird. In der index.html werden auch alle globalen Objekte wie die WebSocket-Verbindung und die Pin-Liste angelegt.\\

Bei einem Seitenaufruf werden zunächst die benötigten Bibliotheken beim Webserver angefordert und geladen. Das Event \textit{document.ready} stößt dann den Verbindungsaufbau zum WebSocket Server an. Ist die Verbindung hergestellt, wird das Event \textit{websocket.onopen} ausgelöst und die aktuelle Pin-Konfiguration vom Server angefordert.

Die weiteren Inhalte werden dynamisch via JavaScript und \gls{html}-Templates generiert.


\subsection{Skriptdokumente}
Die Funktionen für die Webseite sind analog zu den Modulen des WebSocket Servers aufgeteilt. Die Skriptdokumente der Webseite sind jeweils via \gls{json} mit einem Pseudo-Namespace versehen. Funktionen und Objekte sind dabei in einer \gls{json}-Struktur angelegt und können darüber abgerufen werden. Auch Sub-Namespaces sind möglich, um beispielsweise Hilfsfunktionen zu beherbergen (Abb. \ref{lst:pseudeNamespaceJS}).

\begin{figure}[ht]
\begin{lstlisting}
var bonescriptCtrl = {};
    bonescriptCtrl.util = {};
\end{lstlisting}
\caption{Pseudo-Namespace in JavaScript}
\label{lst:pseudeNamespaceJS}
\end{figure}

Da es innerhalb des Browsers keine modulare Struktur wie im WebSocket Server gibt und um daher keinen falschen Eindruck zu erwecken, sind im Folgenden alle Skriptdokumente alphabetisch zusammengestellt.

\begin{longtabu} to \textwidth {
X[1]
X[3]}
\textbf{bonescriptCtrl.js} & Hier werden alle Funktionen zum Absetzen von Steuerbefehlen an den WebSocket Server definiert. Sowohl Befehle für die Hardware-Steuerung als auch für das Interface sind hinterlegt.\newline\\

\textbf{diagramCtrl.js} & Enthält Funktionen, um die Diagramme zu verwalten. Zum einen werden hierüber die Diagramme initialisiert, zum anderen Funktionen zum Hinzufügen neuer Wertepaare und zum Zurücksetzten des Diagramms registriert. Die Zeichengeschwindigkeit ist fest auf 25Hz gesetzt, um die Systemlast möglichst gering zu halten.\newline\\

\textbf{init.js} & Dieses Dokument liefert eine Funktion, \textit{init()}, in der der eigentliche Inhalt der Seite generiert wird. Sobald die Pin-Liste übermittelt wurde, arbeitet eine For-Each-Schleife jeden einzelnen Pin ab. Dabei werden via synchronem Ajax Request die Templates vom Webserver angefordert. Da die Dokumente zunächst im Cache des Browsers verbleiben, entsteht durch den häufigen Aufruf kein erhöhter Netzwerkverkehr. Im Anschluss werden die Listener-Funktionen an Buttons, etc. angehängt.\newline\\

\textbf{responseHandler.js} & Der Response Handler ist eine Sammlung von Funktionen, um auf eingehende Messages zu reagieren. Zu jedem im WebSocket Server definierten \textit{type} gibt es hier eine entsprechende Funktion. Funktionsnamen sind identisch, damit die Fuktionen direkt aufgerufen werden können.\newline\\

\textbf{util.js} & Dieses Dokument enthält einige Utility-Funktionen, zur einfacheren Bedienung von Bootstrap.\newline\\

\textbf{websocketCtrl.js} & Hier ligen die Listener-Funktionen für die WebSocket-Verbindung. Bei einem erfolgreichen Verbindungsaufbau wird der Buttom im Hautfenster verändert und der Init-Prozess angestoßen. Eingehende Messages werden in \gls{json}-Objekte übersetzt und die entsprechende Funktion aus dem Response Handler aufgerufen.\newline\\
\end{longtabu}


\subsection{Bibliotheken}

\subsubsection{jQuery}
jQuery ist ein \gls{de-facto-standard} zur \gls{dom}-Verwaltung. Diese Bibliothek ermöglicht objektähnliche Verwendung von \gls{html}-Elementen und stellt viele Render- und Animationsfunktionen zur Verfügung.\\

jQuery ist Vorraussetzung für alle weiteren Bibliotheken.


\subsubsection{Bootstrap \& jQuery-UI}
Bootstrap ist für die graphische Darstellung verantwortlich und wird durch einzelne Funktionen aus jQuery-UI ergänzt. Hierbei ermöglicht dieses Framework ein Kachelsystem, das zur Anordnung der Bedienelemente verwendet wird. Dieses System ermöglicht eine dynamische Darstellung, sodass deaktivierte Bedienelemente nicht einfach nur leere Felder hinterlassen. Die aktiven Felder werden vielmehr übersichtlich zusammen gerückt. Weiter generiert Bootstrap mit Hilfe von Themes alle Elemente wie Kopfleiste, Buttons oder Eingabemasken.\\


\subsubsection{Flot Diagrams}
Diese Bibliothek dient zum Zeichnen der Diagramme. Im Dokument wird hierfür nur ein Platzhalter eingetragen. Alles weitere ist Aufgabe der Bibliothek. Um neue Messpunkte einzutragen, wird zunächst das Array, welches alle darzustellenden Wertepaare enthält, aktualitisiert und per Funktionsaufruf neu gerendert.


\subsubsection{Mustache.js}
Mustache.js ist eine JavaScript Implementierung des \href{http://mustache.github.io/}{Mustache-Template-Systems}. Mit diesem Werkzeug lassen sich ohne großen Aufwand HTML-Templates schreiben, bei denen Variablen in doppelten geschweiften Klammern -- \{\{variable\}\} -- eingesetzt werden. Diese werden in der Laufzeit mittels \textit{Mustache.render()} zu einem gültigen HTML-String verarbeitet. Da sich die einzelnen Kacheln nur in der Pin-Bezeichnung unterscheiden, wurde für jeden der drei Kacheltypen jeweils ein Template entworfen.
\chapter{Erweiterungsmöglichkeiten}
Im folgenden Kapitel werden weitere Möglichkeiten erläutert, durch die der Boneserver noch besser an bestehende Laborumgebungen angepasst und weiteres Potential des Konzeptes nutzbar gemacht werden kann.

\section{Zusätzliche Anpassungsoptionen}
Das Interface könnte durch zusätzliche Anpassungsoptionen erweitert werden. Eine Mög\-lich\-keit besteht darin, die einzelnen Kacheln mit individuellen Bezeichnungen zu versehen.

Der \textit{Interface Controller} kann dahingehend erweitert werden. Da im WebSocket Server die Pink-Konfiguration ohnehin in einer \gls{json}-Struktur gespeichet wird, können eventuelle Namen einfach hier gespeichert werden. Die Daten werden im Initialisierungsprozess der Seite automatisch an den Browser übertragen.

Im \textit{Init Script} kann die Weiterverarbeitung ähnlich dem Status der Kacheln implementiert werden (Abb. \ref{lst:fakeResponseMessage}).

\begin{figure}[H]
\begin{lstlisting}
for(pin in pins) {
  if (!pins[pin].active) {
    responseHandler.toggle({parameters: {pin: pin}, response: false});
  }
}
\end{lstlisting}
\caption{Fake response message um geladene Kacheln vo verändern}
\label{lst:fakeResponseMessage}
\end{figure}


\section{Höher aufgelöste A/D-Wandlung}
Der maximale Takt der digitalen Inputs und der \gls{adc} ist durch die kleinste Zeiteinheit in JavaScript (1ms) vorgegeben. Denkbar wäre es, eine höher aufgelöste Taktung in einer separaten Software zu implementieren. In C/C++ könnte eine wesentlich höhere Ab\-tast\-rate erreicht werden. Diese Samples können dann zu mehreren als Antwort übertragen werden.\\

Der \textit{Diagram Controller} muss dahingehend erweitert werden, nicht nur einzelne Wertepaare zu verarbeiten (Abb. \ref{lst:insertTupel}), sondern oben beschriebene Frames. Je nach Abtastrate sollte auch über eine entsprechende Unterabtstung für die Anzeige nachgedacht werden.

\begin{figure}[H]
\begin{lstlisting}
diagramCtrl.util.addValue = function(pin, data) {
  diagramCtrl[pin]['data'].push(data);
}
\end{lstlisting}
\caption{Tupel in ein Diagramm einfügen}
\label{lst:insertTupel}
\end{figure}


\section{Alternative Steuerungsbibliothek -- octalbonescript}
Die Bibliothek \textit{octalbonescript}\footnote{https://github.com/theoctal/octalbonescript}, ein bonescript Fork, löst laut API verschiedene Probleme der Bonescript-Bibliothek. Sie könnte als Ersatz für die bisher verwendete Bonescript-Bibliothek genutzt werden. Dadurch würden auch die Module \textit{gpioControl.js} und \textit{pwmControl.js} nicht mehr nötig sein, da sie nur Bypass-Funktionen für fehlerhafte Funktionen enthalten.


\section{Alternative Hardware}
Denkbar wäre eine Portierung des Boneservers auf eine andere Hardware-Plattform. Die Voraussetzung hierfür ist ein Linux-System mit einem Netzwerkzugang und entsprechendem Speicher.\\

Um auf einer anderen Hardware arbeiten zu können, ist eine passende Bibliothek wie die Bonescript-Bibliothek BeagleBone, nötig, um die Hardware zu steuern. Das Modul \textit{boneControl.js} regelt die gesamte Steuerung und muss dahingehend erweitert werden, dass die enthaltenen Fälle (Abb. \ref{lst:exampleCase}) passende Funktionen aus ihrer Bibliothek starten.

\begin{figure}[H]
\begin{lstlisting}
case 'analogRead':
  var pin = parameters.pin;

  response = bonescript[request.type](pin);
  break;
\end{lstlisting}
\caption{Beispiel-Case aus dem Modul \textit{boneControl.js}}
\label{lst:exampleCase}
\end{figure}


\section{Erweiterte Konfiguration in das Interface integrieren}
Die erweiterten Konfigurationsmöglichkeiten, die bisher aus Sicherheitsgründen nur über eine Konfigurationsdatei möglich waren, können in das Interface integriert werden. Das Authentifizierungssystem bietet hierfür die Möglichkeit, verschiedene Benutzer- und Ad\-mi\-ni\-stra\-tor-""Accounts einzurichten (vgl. Abb. \ref{lst:lighttpdhtdigest}). Dazu muss eine Administrationsseite geschrieben und in einem passenden Verzeichnis abgelegt werden. Die Zugriffsbeschränkungen können dann, wie oben beschrieben, separat für dieses Verzeichnis konfiguriert werden. Externe Laufwerke und USB-Sticks sollten dafür automatisch mit passenden Rechten eingebunden werden und in einem Menü auswählbar sein.


\section{Weitere GPIO}
Neben den bereits verfügbaren GPIO sind noch einige weitere vorhanden, die standardmäßig für den HDMI-Ausgang verwendet werden. Da das System primär per Fernzugriff verwaltet werden soll, kann der Anschluss deaktiviert werden. Diese GPIO können über die \textit{whitelist.json} freigegeben werden.
\chapter{Zusammenfassung}
Im Rahmen dieser Arbeit wurde von mir ein webbasiertes Steuersystem für Messanwendungen entwickelt und vorgestellt. Die Grundidee war, ein vereinfachtes, flexibles System anzubieten, das je nach Anwendungskontext individuell und ohne großen Aufwand angepasst werden kann. Hardware und Programmierung sollten hauptsächlich mit auf dem Markt vorhandenen Mitteln realisiert werden, um Entwicklungsaufwand und Kosten gering zu halten. Auch Rechenleistung und Energie sollten überschaubar bleiben. Von zentraler Bedeutung für das Messsystem allerdings ist, dass es über das Internet, abruf- und steuerbar ist.

Die Wahl der Hardware fiel auf den BeagleBone Black, weil bei ihm das Verhältnis zwischen Rechenleistung, Schnittstellenumfang und Preis gegenüber den Konkurrenzsystemen am günstigsten ist. Gleichzeitig ist der Prozessor eine aktive Produktlinie von Texas Instruments, einem namhaften Hersteller für Mikroprozessoren und es gibt bereits werksseitig eine Software-Bibliothek, die die rudimentäre Hardware-Steuerung übernimmt. Die Wahl des Betriebssystems fiel auf Arch Linux, ein minimales Linux System mit großen Anpassungsmöglichkeiten, da eine Eigenentwicklung den Rahmen dieser Arbeit bei weitem gesprengt hätte und Linux Systeme im Embedded-Bereich zurzeit einen De-Facto-Standard darstellen.\\

Da die Bibliothek zu steuerung der Hardware in JavaScript bzw. Node.js implemtiert ist, habe ich zunächst versucht, möglichst viel der benötigten Funktionalität via JavaScript umzusetzen. Größtes Problem dabei war, dass die meisten Webserver-Lösungen in Node.js im Zusammenhang mit dem BeagleBone viel zu langsam arbeiten. Zudem ist es sehr aufwändig,  die vollständige Funktionalität eines Webservers selbst zu implementieren und das gesamte System hätte mit root-Rechten laufen müssen, was bei Webservern aus Sicherheitsgründen grundsätzlich vermieden werden sollte. Hier wäre weiterer Aufwand durch gewissenhaftes Implementieren von Sicherheitsmechanismen entstanden. Dies war allerdings nicht Teil des Projektes. Aus diesen Gründen habe ich mich für ein mehrstufiges System entschieden, bei dem ein regulärer Webserver Dokumente ausliefert und ein zweites System sich ausschließlich mit der Verwaltung der GPIO beschäftigt.

Dabei stellte sich die Frage, in welcher Weise Steuerbefehle sinnvoll an den BeagleBone übertragen und Antworten bzw. Messdaten zurück erhalten werden könnten. Die Lösung dieses Problems fand sich in WebSockets und \gls{json}. Mit deren Hilfe ist es möglich Daten ohne großen Daten-Overhead strukturiert zu auszutauschen. Vorteilhaft ist ebenfalls, dass es bereits einige Implementierungen in Node.js in Form von Modulen gibt.\\

Das größte Problem wärend der Entwicklung bestand darin, dass die Bonescript-Bibliothek durch ihr frühes Entwicklungsstadium noch einige Fehler insbesondere in der Steuerung der Pulsbreitenmodulation aufweist. Lange Versuche diese Fehler zu beheben oder zu umgehen haben dazu geführt, dass ich die Funktionalität selbst in Modulform separat implementiert habe. Ziel ist es aber dennoch, diese Funktionaltät wieder in die Bibliothek zu verlagern, sobald die Probleme dort gelöst sind. 

Gegen Ende des Projektes bin ich auf eine alternartive Bibliothek gestoßen (\textit{octalbonescript}), die ihrer Beschreibung nach die API der Bonescript-Bibliothek vollständig implementiert, dabei allerdings wesentliche Fehler behebt. Im Rahmen dieser Arbeit blieb leider keine Zeit diese vielversprechende Alternative eingehend zu prüfen.\\

Das Ergebnis meiner Arbeit ist ein System, mit dem sich einfache Messumgebungen schnell erstellen lassen und das außer einer Netzwerkverbindung keiner weiteren Technik bedarf. Dabei ist es unerheblich, ob es sich bei dem Netzwerk um ein lokales Netzwerk, ein Intranet im Laborkoplex oder das Internet handelt. Da nur wichtig ist, dass das Netzwerk Betribssystemweit vorhanden ist, sind auch GSM- oder WLAN-Verbindungen kein Problem.

Diese Arbeit zeigt, dass es mit Hilfe von Webtechnologien wie HTML5 und WebSockets möglich ist Echtzeitmesssysteme zu realisieren, die existierende Hardware verwenden und dadurch wesentlich kostengünstiger sind als spezialisierte Systeme namhafter Hersteller.


\appendix
\part{Anhang}
\chapter{Betriebsanleitung}
\input{manual/1-hardware.tex}
\section{Installation}
Als Betriebssystem wird \href{http://archlinuxarm.org/}{Arch Linux ARM} verwendet, eine Portierung von Arch Linux für ARM-Prozessoren. Arch Linux ARM stellt auch ein spezielles package repository zur Verfügung.

\subsection{SD-Karte vorbereiten}
Auf der Homepage von Arch Linux ARM gibt es eine Installationsanleitung, die laufend aktualisiert wird. Die folgende Anleitung ist daher im Wesentlichen eine Übersetzung ausgehend von einem Linux als Host-System. Stattdessen kann auch die mitgelieferte {\AA}ngström Distribution verwendet werden, die mit dem BeagleBone ausgeliefert wird.

\paragraph{Voraussetzungen} sind die Softwarepakete \textit{dosfstools} und \textit{wget} sowie root-Rechte und eine Micro SD-Karte mit mindestens 2GB Speicherkapazität.

\begin{enumerate}

\item Finden Sie zunächst heraus, welcher Laufwerkspfad der vorgesehenen SD-Karte enspricht. Meist \textit{/dev/sd[a, b, ...]} oder \textit{/dev/mmcblk[0, 1, ...]}.

{\paragraph{\color{red} ACHTUNG} \textbf{\color{red} Überprüfen Sie Laufwerkspfade genau, bevor Sie mit der Installation beginnen, da sonst irreparable Schäden am Host-System auftreten können!}}

\item Starten Sie \emph{fdisk}, um die SD-Karte zu formatieren:

\begin{lstlisting}
fdisk /dev/sdX
\end{lstlisting}

\item Erstellen Sie eine neue Partitionstabelle und die nötigen Partitionen.\\
Dazu geben Sie nacheinander die folgenden Kommandos ein (jeweils mit \textit{enter} bestätigen):

\begin{longtabu} to \textwidth {
	X[1]
    X[3]}
\textbf{Kommando} & \textbf{Funktion}\\
o & Erzeugt eine neue Partitionstabelle\\
n, p, 1 & Erzeugt eine neue (n), primäre (p), erste (1) Partition\\
\textit{enter} & Bestätigt den Default-Wert für den ersten Sektor\\
+64M & +64M als letzten Sektor setzt die Partitionsgröße auf 64MByte\\
t, e & Ändert den Partitionstyp auf \textit{W95 FAT16 (LBA)}\\
a, 1 & Setzt das boot flag der ersten Partition (je nach \emph{fdisk}-Version wird die erste Partition automatisch ausgewählt, da nur eine zur Verfügung steht)\\
n, p, 2 & Erzeugt eine neue (n), primäre (p), zweite (2) Partition\\
2x \textit{enter} & Setzt die Default-Werte für den ersten und letzten Sektor der Partition\\
w & Schreibt Änderungen in die Partitonstabelle
\end{longtabu}

\item Formatieren der ersten Partition:

\begin{lstlisting}
mkfs.vfat -F 16 /dev/sdX1
\end{lstlisting}

\item Formatieren der zweiten Partition:

\begin{lstlisting}
mkfs.ext4 /dev/sdX2
\end{lstlisting}

\item Laden Sie den \textit{bootloader tarball} herunter und entpacken Sie ihn auf die erste Partition der SD-Karte:

\begin{lstlisting}
wget http://archlinuxarm.org/os/omap/BeagleBone-bootloader.tar.gz
mkdir boot
mount /dev/sdX1 boot
tar -xvf BeagleBone-bootloader.tar.gz -C boot
sync && umount boot
\end{lstlisting}

\item Laden Sie den \textit{rootfs tarball} herunter und entpacken Sie ihn auf die zweite Partition der SD-Karte (hierzu müssen Sie als \textit{root} eingeloggt sein, \emph{sudo} reicht in diesem Fall nicht):

\begin{lstlisting}
wget http://archlinuxarm.org/os/ArchLinuxARM-am33x-latest.tar.gz
mkdir root
mount /dev/sdX2 root
tar -xf ArchLinuxARM-am33x-latest.tar.gz -C root
sync && umount root
\end{lstlisting}

\item Stecken Sie die SD-Karte in den BeagleBone und halten Sie die Taste S2 gedrückt, um von der SD-Karte zu booten, während Sie die Power-Taste (S3) betätigen.\\
Wenn das System gestartet ist, können Sie sich auf der Kommandozeile oder via \emph{ssh} einloggen.\\

Benutzernahme/Passwort lautet \textbf{root/root}.

\end{enumerate}

Aus Sicherheitsgründen sollten Sie nach dem Systemstart als erstes das root-Passwort ändern:

\begin{lstlisting}
passwd root
\end{lstlisting}

Da man sich, außer zu Wartungszwecken, nicht am System anmelden muss, kann auf die Erstellung eines regulären Benutzers verzichtet werden.


\subsection{Installation im internen Speicher}

\paragraph{Hinweis} Der BeagleBone hat zwar eine eingebaute Uhr allerdings keine Batterie. Nach einem Neustart kann es daher passieren, dass die interne Uhr auf den Default-Wert zurück gesetzt wird. Überprüfen Sie mittels \emph{date} die aktuelle Systemzeit und aktualisieren diese gegebenenfalls via \texttt{ntpdate -u pool.ntp.org}

\begin{enumerate}
\item Um Arch Linux direkt auf der eMMC zu installieren, aktualisieren Sie zunächst das eben gestartete System und installieren die Pakete \emph{wget}, \emph{dosfstools} und \emph{ntp}.

\begin{lstlisting}
pacman -Syu wget dosfstools ntp
\end{lstlisting}

Das Paket \emph{ntp} stellt hierbei das Programm \emph{ntpdate} zur Verfügung (s. O.).

\item Der interne Speicher ist bereits korrekt partitioniert, folgen Sie daher nur den Schritten 4 bis 7. Die Partionen sind \textit{mmcblk1p1} bzw. \textit{mmcblk1p2} (s. O.).

\item Fahren Sie das System herunter und warten Sie bis alle LEDs erloschen sind.

\item Entfernen Sie die SD-Karte und starten das System erneut.
\end{enumerate}


\subsection{boneserver installieren}

\paragraph{Repository klonen}
\textit{boneserver} ist via \href{https://github.com/XMrVertigoX}{GitHub}\footnote{https://github.com/XMrVertigoX} verfügbar. Führen Sie dazu zunächt ein Systemupdate durch, um alle Pakete auf den neusten Stand zu bringen und installieren Sie das Paket \emph{git}. Anschließend klonen Sie das Repository nach \texttt{/opt}.

\begin{lstlisting}
pacman -Syu git
git -C /opt clone https://github.com/XMrVertigoX/boneserver.git
\end{lstlisting}

\paragraph{Installationsskript ausführen} Im root-Verzeichnis des Repositories befindet sich ein Skript, welches die weitere Installtaion übernimmt. Wechseln Sie dazu in das Verzeichniss und führen das Installationsskript aus.

\begin{lstlisting}
cd /opt/boneserver
./install.sh
\end{lstlisting}

Hierbei werden alle erforderlichen Pakete und Module installiert, die Konfigurationsdateien verlinkt sowie die Daemons installiert und gestartet.\\

Wenn das Skript fehlerfrei durchgelaufen ist, wird der BeagleBone automatisch neu gestartet und die Installation ist abgeschlossen.

\section{Betrieb}
Für die Verwendung des Webinterface wird eine Netzwerkverbindung zum BeagleBone und ein Webbrowser\footnote{Das Webinterface verwendet \emph{JQuery} in der Version 2.1.1, aktuelle webbrowser sollten hier keine Probleme bereiten. Ansonsten kann die Homepage von JQuery konsultiert werden: \url{http://jquery.com/browser-support/}} mit aktiviertem JavaScript vorausgesetzt.


\subsection{Netzwerkverbindung herstellen}

\paragraph{Hinweis:} Die Standardkonfiguration des BeagleBone sieht den Betrieb mit einem DHCP-Server vor. Sollte dies nicht gewünscht oder möglich sein, kann über die üblichen Wege eine statische IP eingestellt werden. Anleitungen hierzu findet man im Internet.\\

Steht ein DNS-Server zur Verfügung, kann der BeagleBone über seinen Hostname erreicht werden, standardmäßig \textit{boneserver}. Ansonsten finden Sie zunächst heraus, welche IP dem BeagleBone zugewiesen wurde. Hierfür kann entweder die Routing-Tabelle des DHCP-Servers konsultiert werden oder in der Kommandozeile via \emph{ip} die aktuelle Addresse der einzelnen Netzwerkadapter abgerufen werden \mbox{(Abb. \ref{fig:getBeagleBoneIP})}.

\begin{figure}[ht]
	\centering
	\includegraphics[width=0.7\textwidth]{manual/images/getBeagleBoneIP.png}
	\caption{IP des BeagleBone abrufen}
	\label{fig:getBeagleBoneIP}
\end{figure}


\subsection{Webinterface aufrufen}
Das Webinterface kann einfach über DNS-Namen oder die IP im Webbrowser aufgerufen werden. Ist die Verbindung hergestellt, wird dies durch einen grünen Haken rechts in der Titelleiste angezeigt \mbox{(Abb. \ref{fig:mainWindowConnected})} und die Steuerelemente werden generiert.
Sollte die Verbindung einmal unterbrochen werden, wechselt dieser Haken in ein rotes Kreuz. In diesem Fall kann die Seite neu geladen werden, eventuelle Konfigurationen bleiben erhalten.

\paragraph{Passwortschutz} Um unbefugten Zugriff zu verhindern ist das Webinterface passwortgeschützt. Wenn Sie dieses Passwort änderen möchten, generieren Sie zunächst einen neuen Datensatz z. B. mit dem \href{http://jesin.tk/tools/htdigest-generator-tool/}{\textit{htdigest Generator Tool}}\footnote{ \url{http://jesin.tk/tools/htdigest-generator-tool/}} und tragen die Zugangsdaten in die Datei \textit{config/lighttpd/lighttpd.user} ein. Die Default Login-Daten sind \textbf{admin/AgG7KgW4}

\begin{figure}[ht] 
	\centering
	\includegraphics[]{manual/images/mainWindowConnected.png}
	\caption{Webinterface verbunden}
	\label{fig:mainWindowConnected}
\end{figure}

\paragraph{Hinweis} Das Webinterface kann immer nur von einem Fenster aus aufgerufen werden. Es kann daher passieren, dass bei einem schnellen Fensterwechsel oder Neuladen der Seite die Verbindung nicht sofort hergestellt wird. In dem Fall einfach ein paar (Milli-)Sekunden warten, bis die Verbindung wieder frei ist.


\subsection{Bedienelemente}
Wie in \mbox{Abbildung \ref{fig:webinterface}} gezeigt hat die Weboberfläche des boneserver drei Anzeigegruppen:

\begin{itemize}
\item[\textbf{1}] \textbf{Verbindungsanzeige}\\
zeigt grün, wenn die Steuereinheit verbunden ist und rot, wenn die Verbindung unterbrochen ist (vgl. Abb. \ref{fig:mainWindowConnected}).
\item[\textbf{2}] \textbf{Bedienfelder} für digitale I/O, PWM und AIN\\
Hier findet die tatsächliche Bedienung der GPIO statt. Es gibt drei Sektionen jeweils für digitale I/O, PWM und AIN. Die Bedienung der verschiedenen Kacheln wird weiter unten beschrieben.
\item[\textbf{3}] \textbf{Anzeigenschalter} für die einzelnen Pins\\
Hier können einzelne Kacheln ein- bzw. ausgeblendet werden, um die Oberfläche übersichtlicher zu gestalten und an die Arbeitsumgebung anzupassen. Diese Funktion dient ausschließlich der Übersicht, eine ausgeblendete Kachel bleibt weiterhin aktiv und kann jederzeit wieder eingeblendet werden. Diese Einstellungen bleiben auch nach einem Neustart erhalten.
\end{itemize}

\begin{figure}[ht] 
	\centering
	\includegraphics[width=1.0\textwidth]{manual/images/controlElements.png}
	\caption{Weboberfläche}
	\label{fig:webinterface}
\end{figure}

\subsubsection{Digitale In- und Outputs}
Die Konfigurationskachel \mbox{(Abb. \ref{fig:mainWindowGPIODisabled})} für digitale I/O besteht aus zwei Schaltern: Betriebsrichtung und logic level.

\paragraph{Betriebsrichtung} Jeder digitale I/O kann entweder als Input oder als Output konfiguriert werden. Dazu kann über den Wahlschalter \textbf{In/Out} jederzeit das Gewüschte ausgewählt werden.

\paragraph{logic level} Wenn der GPIO als Output konfiguriert ist, kann hier mittels der beiden Schaltflächen, \textbf{1} und \textbf{0}, ein logisches high und ein logisches low eingestellt werden. Ist der GPIO als Input konfiguriert, ist diese Schaltfläche deaktiviert und zeigt stattdessen den Status der Leitung an. Die GPIO sind mit einem internen Pulldown-Widerstand beschaltet.

\begin{figure}[ht] 
	\centering
	\includegraphics[]{manual/images/mainWindowGPIODisabled.png}
	\caption{Deaktivierte GPIO-Kachel}
	\label{fig:mainWindowGPIODisabled}
\end{figure}


\subsubsection{Pulsbreitenmodulation (PWM)}
Mit Hilfe der PWM-Kacheln \mbox{(Abb. \ref{fig:mainWindowPWMDisabled})} werden die GPIO konfiguriert, über die eine Pulsbreitenmodulation möglich ist.\\

Der BeagleBone stellt insgesamt sieben PWM-Ausgänge mit insgesamt vier Timern zur Verfügung. Dabei teilen sich jeweils die Pinne P8\_13/19, P9\_14/16 und P9\_21/22 einen Timer. P9\_42 hat einen exklusiven Timer. Die Ausgänge mit einem gemeinsamen Timer haben dementsprechend immer dieselbe Frequenz und laufen absolut synchron.\\

Über die Buttons \textbf{Enable} und \textbf{Disable} kann der jeweilige Ausgang aktiviert bzw. deaktiviert werden. Wenn ein PWM-Ausgang deaktiviert wird, werden alle Einstellungen bezüglich Frequenz und Pulsbreite gelöscht!

\paragraph{Periodendauer} Über das Eingabefeld \textbf{Period} wird die Periodendauer in Nanosekunden (ns) eingestellt. Kleinster Wert ist hier 1 ns ($\hat{=}$ 1GHz) und der größte $10^9$ ns (= 1s $\hat{=}$ 1Hz).

\paragraph{Pulsbreite} Über das Eingabefeld \textbf{Duty} wird die Pulsbreite zwischen 0 und 1 eingestellt. Hier wird intern ebenfalls mit Nanosekunden gearbeitet, daher kann der tatsächliche Wert zusätzliche Nachkommastellen bekommen.

\begin{figure}[ht] 
	\centering
	\includegraphics[]{manual/images/mainWindowPWMDisabled.png}
	\caption{Deaktivierte PWM-Kachel}
	\label{fig:mainWindowPWMDisabled}
\end{figure}

Mit \textbf{Write} werden die Parameter abgesendet.

\paragraph{Hinweis} Der Linux Kernel arbeitet intern mit ganzen Nanosekunden, daher ist die Genauigkeit der Pulsbreite von der Höhe der Periodendauer anhängig.

\paragraph{Bug:} Wenn beide Ausgänge eines PWM-Generators aktiviert sind, lässt sich die Frequenz bei keinem der beiden ändern. Wenn bei einem der beiden PWMs die Frequenz zunächst geändert wurde, kann der zweite Ausgang nicht verwendet werden. Dies ist ein Problem der im Hintergrund verwendeten device tree overlays und wird in nachfolgenden Versionen der Bibliothek behoben.


\subsubsection{Analoge Inputs}
Mit diesen Kacheln \mbox{(Abb. \ref{fig:mainWindowADC})} werden die Analog/Digital-Converter gesteuert und die Eingangswerte in einem Echtzeitdiagramm angezeigt.\\

Mit den Tasten \textbf{Start} und \textbf{Stop} wird die Aufzeichnung gestartet bzw. gestoppt. Parallel zur Anzeige werden die Messdaten aufgezeichnet. Über \textbf{Download} können sie als CSV-Datei heruntergeladen werden.\\

Die Taste \textbf{Delete} löscht die zu diesem Eingang gespeicherte Messreihe um Speicherplatz frei zu machen.

\paragraph{\color{red} ACHTUNG} \textbf{\color{red} Laden sie Messreihen immer herunter, bevor Sie sie löschen, die Messdaten sind sonst unwiederbringlich verloren!}

\begin{figure}[ht] 
	\centering
	\includegraphics[width=0.9\textwidth]{manual/images/mainWindowAIN.png}
	\caption{ADC-Kachel}
	\label{fig:mainWindowADC}
\end{figure}


\subsection{Erweiterte Einstellungen}
Zusätzlich gibt es die Möglichkeit, über die Datei \textit{settings.json} im root-Verzeichnis des boneserver weitere Einstellungen vorzunehmen. Ist keine solche Datei vorhanden, wird die mitgelieferte \textit{settings-default.json} verwendet. Die Parameter überschreiben die Default-Werte, es müssen daher nur abweichende Werte eingetragen werden.\\

Die Datei ist eine JSON\footnote{JavaScript Object Notation (JSON) ist in der RFC 7159 beschrieben}-Datei in der folgende Parameter eingestellt werden können:\\

\begin{longtabu} to \textwidth {X[1] X[3]}
    
  \textbf{host} & IP des WebSocket Servers\newline
  Dieser Wert sollte nicht verändert werden, da sonst der WebSocket Server möglicherweise nicht mehr über das Webinterface erreichbar ist.\newline
  \textit{default: localhost}\\
  \textbf{port} & Netzwerk-Port des WebSocket Servers\newline
  Dieser Wert sollte nicht verändert werden, da der WebSocket Server sonst nicht mehr über das Webinterface erreichbar ist.\newline
  \textit{default: 8081}\\
  \textbf{gpioSampleRate} & Abtastrate der digitalen Inputs in Millisekunden\newline
  Angegeben wird die Zeit zwischen den Abfragen. Dieser Wert kann erhöht werden um die System- und Netzwerklast zu verringern.\newline
  \textit{default: 100}\\
  \textbf{adcSampleRate} & Abtastrate der analogen Inputs in Millisekunden\newline
  Angegeben wird die Zeit zwischen den Abfragen. Dieser Wert kann erhöht werden, um die System- und Netzwerklast sowie um die Menge der erhobenen Messwerte zu verringern. Dies ermöglicht bei gleichem Speicherplatz längere Messreihen.\newline
  \textit{default: 10}\\
  \textbf{dataLocation} & Speicherpfad für die Messdaten der ADC\newline
  Hier kann ein alternativer Pfad zur Speicherung der Messdaten eintragen werden. Es können auch externe Speicherorte wie z. B. USB-Sticks, USB-Festplatten oder Netzlaufwerke verwendet werden.\newline
  \textit{default: ./data}
\end{longtabu}

\paragraph{Hinweis} Die Datei \textit{settings-default.json} sollte nicht verändert werden, da sonst nicht ohne weiteres ein Update durchgfeührt werden kann.

\input{manual/4-wartung.tex}

\chapter{Tabellen}

\begin{longtabu} to \textwidth {
  X[1]
  X[4]}
  \textbf{Paketname} & \textbf{Kurzbeschreibung}\\
  base-devel & Enthält die benötigten Entwicklertools\\
  python2 & Python in der Version 2. Wird für den Node Package Manager benötigt.\\
  lighttpd  & Der Webserver (siehe Abschnitt \ref{subsec:Lighttpd}).\\
  vsftpd & Ein FTP-Server. Wird verwendet um Messdaten ohne Webinterface automatisch abzurufen (siehe Abschnitt \ref{subsec:vsftpd}).\\
  linux-headers-am33x-legacy & Linux Header Files. Werden benötigt um Device Tree Overlays zu kompilieren.\\
  nodejs & JavaScript Engine (Siehe Abschnitt \ref{subsec:Node.js}).\\
  wget & Freies Kommandozeilenprogram um Dateien aus dem Internet herunterladen zu können.\\
  zsh & Alternative Kommandozeile (Z-Shell). Hohe Anpassungsmöglichkeiten.\\
  grml-zsh-config & Konfigurationsdateien für zsh.\\
  wpa\_supplicant & Unterstützung für WPA-Verschlüsselte WLAN.\\
  pv & Pipe Viewer. Wird von den Skripten \textit{backup.sh} und \textit{restpre.sh} verwendet um Speicherabbilder des BeagleBone zu erstellen.\\
  dtc-git-patched-20130410-1 & Device Tree Compiler mit \textit{dynamic link}-Patch. Wird von der bonescript library benötigt.\\
  haproxy-1.5.3-1 & Der Proxy-Server (siehe Abschnitt \ref{subsec:HAProxy}).\\
  
  \caption{Liste der benötigten Zusatzpakete}\\
  \label{tab:additionalPackages}\\
\end{longtabu}

\printglossary[title={Glossar}]
\bibliography{appendix/literature.bib}
\addchap{Eidesstattliche Erklärung}

\noindent Ich versichere hiermit, die vorgelegte Arbeit in dem gemeldeten Zeitraum ohne fremde Hilfe verfasst und mich keiner anderen als der angegebenen Hilfsmittel und Quellen bedient zu haben.

\vfill

\noindent Köln, den \today \\[1cm]

\noindent Caspar Friedrich


\end{document}